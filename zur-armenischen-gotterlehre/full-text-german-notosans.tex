%XeLaTeX
\documentclass{article}
\setlength{\emergencystretch}{10pt}
\usepackage{microtype}
\usepackage{svg}
\usepackage{float}

%define custom symbols
\newcommand*\svgAAAA{\includesvg[height=1em]{arabic001.svg}}

\usepackage{fontspec}
\usepackage{polyglossia}

\setdefaultlanguage{german}
\setotherlanguages{armenian,greek}

% Set main font (used for German, the default language)
\newfontfamily{\germanfont}{Noto Sans}

% Set specific fonts for other languages
\newfontfamily{\greekfont}{Fira Sans}
\newfontfamily{\armenianfont}{DejaVu Sans}

%\newfontfamily{\arm}[Script=Armenian]{DejaVuSans-Bold}
%\newfontfamily{\armitalic}[Script=Armenian]{DejaVuSans-BoldOblique}
\defaultfontfeatures{Scale=MatchLowercase}
\def\arraystretch{1.15}
\begin{document}
\begin{titlepage} % Suppresses headers and footers on the title page
	\centering % Centre everything on the title page
	%\scshape % Use small caps for all text on the title page

	%————————————————
	%	Title
	%————————————————

	\rule{\textwidth}{1.6pt}\vspace*{-\baselineskip}\vspace*{2pt} % Thick horizontal rule
	\rule{\textwidth}{0.4pt} % Thin horizontal rule
	
	\vspace{1\baselineskip} % Whitespace above the title
	
	{\scshape\Huge Zur armenischen Götterlehre}
	
	\vspace{1\baselineskip} % Whitespace above the title

	\rule{\textwidth}{0.4pt}\vspace*{-\baselineskip}\vspace{3.2pt} % Thin horizontal rule
	\rule{\textwidth}{1.6pt} % Thick horizontal rule
	
	\vspace{1\baselineskip} % Whitespace after the title block
	
	%————————————————
	%	Subtitle
	%————————————————
	
        {\scshape Von \large Heinrich Gelzer}
 
        \vspace{1.0\baselineskip}

‌        {\scshape\small Berichte über die Verhandlungen der königlich sächsischen Gesellschaft der Wissenschaften zu Leipzig}

	%————————————————
	%	Editor(s)
	%————————————————
        \vspace*{\fill}    

        \vspace{1.0\baselineskip}

        {\scshape Philologisch-historische Klasse}
        
	\vspace{1\baselineskip}

        {\scshape\small Leipzig 1896}
		
	\vspace{0.25\baselineskip} % Whitespace after the title block

        {\scshape\small Solar Anamnesis Edition}% Publication year}
    
	{\scshape\footnotesize CC0 1.0 Universal } % Publisher
\end{titlepage}
\clearpage
\tableofcontents
\clearpage
\section*{}
\begin{center}
\Large
\textbf{Sitzung am 7. Dezember 1895.}

\textbf{Herr H. Gelzer hielt einen Vortrag: \emph{Zur armenischen Götterlehre}.}
\end{center}
\paragraph{}
Die Quellen, aus denen wir unsre Kenntnis der armenischen Mythologie schöpfen, sind leider außerordentlich dürftig. Lieder eigentlich mythologischen Inhalts sind uns nur sehr spärlich erhalten aus nahe liegenden Gründen. Gerade gegen diese Lieder musste sich, wie im fränkischen Reiche, die Feindschaft der Geistlichkeit mit ganzer Energie und auch mit völligstem Rechte wenden; denn so lange diese geliebten alten Gesänge von den väterlichen Göttern das Volksbewusstsein beherrschten, konnte von einer Durchdringung desselben mit einem wahrhaft christlichen Geiste keine Rede sein.\footnote{Faustus 3., 13.} Unsre Hauptquelle für das armenische Heidentum ist darum der große Rechenschaftsbericht über die Aufhebung der alten Kulte und die Zerstörung ihrer Heimstätten, das von Agathangelos verfasste Leben des h. Gregors. Werthvolle Überreste der alten Göttersagen finden sich auch bei Mar Abas Katinā und bei Moses von Chorēn. Endlich hat uns einen interessanten Mythus der Astronom und Mathematiker Anania von Širak erhalten.

Die Götter eines Volkes sind die Repräsentanten der verschiedenen historischen Phasen, welche das Volksbewusstsein durchmacht. Es ist kein Paradoxon, sondern eine der größten Wahrheiten der Geschichte, dass das nationale Leben eines Volkes sich im Kampfe ethnischer Gegensätze entwickelt. In Armenien sind, seit das Volk der Hayk͑ seine historischen Wohnsitze bezogen hat, erst iranische, dann syrische und hellenistische Einflüsse dominierend gewesen. Dazu hat bereits im 3. Jahrhundert unsrer Zeitrechnung das Christentum eine vollkommen neue Welt- und Lebensanschauung in die Hochtäler des Araxes, des Euphrat und des Tigris getragen. Aber gerade in der Wechselwirkung dieser Gegensätze hat sich das armenische Nationalbewusstsein gekräftigt und entwickelt, und im Kampfe mit den nationalfremden Kulturelementen ist das patriotische, echt armenische Empfinden zur schönsten Blüthe gelangt.

Eine eingehende Betrachtung des armenischen Pantheons, wie es etwa in der Periode von Tigranes, dem Könige der Könige, bis auf Chosrov den Großen als allgemeine Volksanschauung Geltung besaß, hat keinen geringen methodischen Werth. Wir können hier die Religionszustände eines in heller historischer Zeit von höherer Kultur noch unberührten, d. h. der Schrift und Literatur völlig entbehrenden Volkes kennen lernen, was auf die Verhältnisse ähnlich situierter Völker helle Schlaglichter wirft.\footnote{Von den bisherigen Bearbeitungen der armenischen Mythologie ist mir Emins Buch weder im russischen Original, noch in der französischen Übersetzung zugänglich gewesen. Indessen die zahlreichen Auszüge bei Langlois gewähren einen hinreichenden, wenn auch nicht gerade erfreulichen Einblick. Vor allem ist zu vergleichen L. Alishan \begin{armenian}հին հաւատք կամ հեթանոսական կրօնք Հայոց\end{armenian}. Der alte Glaube oder die heidnische Religion Armeniens. Venedig 1895. Minas Tchéraz in seinen notes sur la mythologie arménienne (transactions of the 9. international congress of Orientalists London 1893 2., S. 822-845) gibt interessante Mittheilungen über den heutigen Volksaberglauben, der natürlich nichts weniger als genuin armenisch ist und mit der alten Religion des Volkes kaum irgendwelchen Zusammenhang noch aufweist. Die Bemerkungen des Verfassers sind teilweise rührend naiv, so erinnert er bei dem recht modernen Geschichtchen von dem kartenspielenden Jüngling, den der Teufel nach Indien entführt, an die aus Indien importierten Götter des Fälschers Zenob von Glak S. 844 u. s. f.}
\clearpage
\section{Der Iranische Einfluss}
\paragraph{}
Die bedeutendste und nachhaltigste Einwirkung hat die armenische Kultur von seiten der iranischen erfahren, sodass man nicht mit Unrecht von einer völligen Iranisierung des Volkes gesprochen hat. Es ist das auch ganz natürlich, da von 66 bis 238 n. Chr. Armenien eine Sekundogenitur der Asien beherrschenden Parther gewesen ist. Allein der iranische Einfluss hat sich fragelos schon in einer viel früheren Epoche geltend gemacht, als erst in der Zeit der Arsakidendynastie. Bereits in altpersischer Zeit, wie aus Xenophons Schilderung hervorgeht, war das auf einer relativ niedrigen Kulturstufe stehende armenische Volk nicht ohne Kenntnis der persischen Sprache. Die von den Königen gegründeten Städte haben schon in der vorarsakidischen Epoche persische Namen. Vgl. Hübschmann: Armenische Grammatik 1. S. 12. Aber der Einfluss der Zentralregierung auf diese Gebirgslandschaften war im Ganzen zweifellos ein äußerst schwacher. Überhaupt wird die kulturelle Bedeutung der Perserzeit gemeinhin überschätzt, weil es die einzige Epoche ist, aus der wir genaueres wissen, da ja die griechischen Nachrichten uns leidlich gut über die vorderasiatischen Zustände vor Alexander unterrichten. Umgekehrt wird die weltgeschichtliche Mission des Parthervolkes erheblich unterschätzt, weil ihre siegreichen Rivalen, die Sāsāniden, ein völlig getrübtes und unwahres Bild von ihrer religiösen Stellung entworfen haben, hauptsächlich auch, weil das Werk des großen Kenners ihres Staats- und Gesellschaftsorganismus, des Poseidonios von Apameia uns verloren gegangen ist. Es ist das große Verdienst von Lagarde\footnote{Vgl. Lagarde, Symmicta S. 33: „Alle die zahlreichen Wörter, welche das Armenische mit dem Neupersischen identisch besitzt, diese alle sind im Armenischen Lehnwörter aus der arsakidischen Zeit und müssen daher dem Stammlande der Arsakiden, Pahlaw angehören, also pahlawī sein. Es können nicht sāsānidische Wörter sein, da die armenische Bibelübersetzung, welche aus der Mitte des 5. Jahrhunderts stammt, sie bereits im gewöhnlichen Gebrauch hat. Zu Lukulls Zeit mögen sie noch nicht den althaikanischen gleichgegolten haben (?); anzunehmen, dass 430 das Haikanische von den den armenischen Arsakiden feindlichen Sāsāniden seit 250 so tief beeinflusst sein sollte, dass jedes zehnte Wort sāsānidisch wäre, dies anzunehmen, sehe ich keine Veranlassung.“} gezeigt zu haben, dass die in Armenien eingedrungene und dort zur Herrschaft gelangte Kultur ganz und völlig parthisch war. Bereits das älteste Literaturdenkmal, die Übersetzung der h. Schrift, zeigt so zahlreiche persische Lehnwörter, dass die Iranisierung Armeniens unbedingt in einer bedeutend älteren Periode stattgefunden haben muss, als die Epoche ist, wo Marzbane des Sāsāniden-Šāhānšāh das Land regierten.

Schon unter dem alten vorneronischen Könighause der Armenier, welches nur die Kataloge des Mar Abas Katinā und des Moses von Chorēn bereits zu einem arsakidischen machen, hat die parthisch-iranische Kultur Armenien stark beeinflusst.\footnote{Schon im 2. vorchristlichen Jahrhundert tragen armenische Fürsten Namen der iranischen Heldensage. J. Marquart, Philolol. N. F. 8. S. 505 N. 91.} Tigranes der Große, der Nachkomme des Reichsgründers Artaxias, hat seine Jugend als Geisel am parthischen Hofe verbracht. Schon vor 66 n. Chr. herrschen mehrfach parthische Prinzen über Armenien; die Sympathien des in politischen Dingen maßgebenden Fürstenadels waren stetsfort auf parthischer Seite. Ohne Frage waren die großen Familien bereits stark iranisiert; daher denn auch späterhin die alteinheimischen Adelsgeschlechter den eingewanderten Pahlaviden (neben den Aršakunik͑ die Karēn, Surēn, Aspahapet u. s. f.) willig den Vorrang vor sich selbst zugestanden. Parthisch ist wohl auch schon zu Tigranes' Zeiten die höfische Sprache in Armenien gewesen.

Diese ganze Iranisierung Armeniens findet nun ihren stärksten Ausdruck in den zahlreichen Göttergestalten, welche die Armenier aus Iran herübergenommen haben.

Hier ist in erster Linie Gott \begin{armenian}Արամազդ\end{armenian} Aramazd zu nennen. Er entspricht völlig dem persischen Ahuramazda. Er heißt „der große und starke Aramazd,“ „der große und starke Aramazd, der Schöpfer Himmels und der Erden.“ In seinem Edikte sagt König Tiridates: „Glück und Heil möge euch zu Theil werden durch die Hülfe der Götter, Fülle der Fruchtbarkeit vom starken Aramazd.“\footnote{Agathangelos Ausg. von Tiflis S. 41 und S. 48: \begin{armenian}զմեծն և զարին Արամազդ, զարարիչն երկնի և երկրի\end{armenian}. S. 83: \begin{armenian}Ողջոյն հասեալ և շինութիւն դիցն օգնականութեամբ, լիութիւն պարարտութիւն յարոյն Արամազդայ\end{armenian}.} Das stimmt genau mit den Glaubensbekenntnissen der Achaemeniden überein z. B. des Xerxes, welcher sagt: „Ein großer Gott ist Auramazda, welcher der größte der Götter ist, welcher diese Erde schuf, welcher jenen Himmel schuf, welcher die Annehmlichkeit schuf für den Menschen.“ Allein andere Äußerungen bei Agathangelos weichen wieder stark von den mazdaistischen Anschauungen ab. Aramazd heißt nämlich: „Gott Aramazd, welcher der Vater aller Götter heißt.“ Mihr heißt sein Sohn; Anahit und Nanēa werden seine Töchter genannt. Ferner lesen wir Agathangelos S. 456: „Er gelangte an den ummauerten Platz Namens Ani, zu den königlichen Stätten der Ruhe, der Gräber der Könige Armeniens. Und dort zerstörten sie auch die Altäre (Idole) des Gottes Aramazd, welcher der Vater aller Götter heißt.“\footnote{Vgl. Spiegel, Erânische Altertumskunde 2. S. 24.} Die Könige, welche göttlichen Geschlechtes sind, ruhen also in der Tempelburg des Aramazd. Das alles ist grundverschieden von dem, was uns als orthodox mazdaistische Glaubenslehre überliefert wird.\footnote{Der Grieche § 132: \begin{greek}τὸν βωμὸν Κρόνου τοῦ πατρὸς Διὸς παντοδαίμονος\end{greek}. Mit Ausnahme dieser Stelle gibt er sonst Aramazd stets durch \begin{greek}Ζεύς\end{greek} wieder. Gutschmid, Kl. Schr. 3. S. 342 erkennt darin das höchste Princip: Zruan. Anders und wohl richtiger Lagarde, Agathang. S. 145.} In Armenien ist die Mazdalehre mit religiösen Vorstellungen einer viel älteren Epoche vermischt. Es findet hier eine Kontamination uralten armenischen Glaubens mit den durch die Parther übermittelten iranischen Vorstellungen statt. Offenbar haben wir uns den Vorgang ähnlich, wie teilweise in Griechenland, zu denken. Wie z. B. auf Rhodos der Kult des \begin{greek}Ζεὺς Ἀταβύριος\end{greek} einfach den phönizischen Baal vom Tabor repräsentiert, welcher sich auf dem Eiland mit dem hellenischen Nationalgott verschmolz, so ist auch in Armenien Aramazd mit dem einheimischen altarmenischen Göttervater zu einem Wesen verbunden worden. Demnach kann auch Aramazd nicht der ursprüngliche Name dieses summus deus gewesen sein, sondern in der vorparthischen Epoche muss er eine andere, für uns verschollene Bezeichnung getragen haben.

Hierher gehört auch Aramazds Sohn \begin{armenian}Միհր\end{armenian} Mihr = Mithras. Sein Hauptheiligtum befand sich in der Provinz Derǰan (Derxene). Agathang. S. 459: Er kam, gelangte zu dem Tempel des Mihr (\begin{armenian}Մրհական մեհեանն\end{armenian} eigentlich: dem mithrischen Tempel) welcher ein Sohn des Aramazd genannt wird, in das Dorf, welches in parthischer Sprache Bagayaṙičn\footnote{\begin{armenian}Բագայառիճն\end{armenian}. Moses Chor. 2., 12 schreibt: \begin{armenian}Բագայառինջ\end{armenian} Bagayaṙinj̑.} heißt.

Ferner ist unter den männlichen Gottheiten iranischen Ursprungs auch \begin{armenian}Սպանդարամետ\end{armenian} Spandaramet zu erwähnen. Im 2. Makkabäerbuche 6. 7 wird Dionysos durch Spandaramet wiedergegeben. Lagarde\footnote{Vgl. ges. Abh. S. 264, 265; armen. Stud. S. 139.} hat darin die zoroastrische Spenta Armaiti wiedererkannt. Wie und wo diese Gottheit in Armenien verehrt wurde, ist gänzlich unbekannt; sie gehört nicht zu den offiziell verehrten Göttern, deren Vernichtung bei dem Siege des Christentums durch die Staatsgewalt geschah. Vielleicht, dass sie den Mittelpunkt irgendeines Lokalkultes bildete. Bei Thomas Arcruni S. 28 (Ausg. v. Petersburg 1887) lesen wir: „Die Erde ist die Herberge der Gottheit Spandaramet, nicht ist sie geschaffen von irgendjemand u. s. f.“ Allein dies ist ein Stück des Glaubensbekenntnisses des Hephthalitischen Königs Manithop, der dort als Lehrer einer zoroastrischen Richtung auftritt und mit Armenien nichts zu tun hat.

Iranischen Ursprungs ist auch der vielleicht nationalste und populärste Gott der Armenier: Vahagn. Bereits Windischmann, dann namentlich Lagarde hat ihn mit dem iranischen Genius des Sieges Verethraghna identificiert.\footnote{Armen. Studien S. 141. Vgl. jetzt besonders: H. Hübschmann, armenische Grammatik 1. S. 75 ff. und die Ausführungen S. 77.} Bei Agathangelos heißt der Gott S. 83 \begin{armenian}Վահագն\end{armenian} Vahagn, wo er als dritter in der Triade der armenischen Landesschutzgottheiten erscheint, ebenso S. 469. Daneben kommt noch (a. a. O. S. 469) zweimal die adjektivische Form \begin{armenian}Վահէվահեան մեհեանն\end{armenian} „der vahēvahische Tempel“ vor. Wir haben also neben Vahagn die Namensform Vahēvahē; wie J. Wackernagel ansprechend vermutet, vielleicht ein orgiastischer Ausruf, eine Kurzform, wie Hymen Hymenaee u. s. f. Bei Agathangelos S. 469 wird nun berichtet: „Als Grigorios in das armenische Gebiet gekommen war, hörte er, dass der Vahēvahische Tempel noch übrig sei im Lande Tarōn, der Tempel, reich an Schätzen, voll Goldes und Silbers; viele Weihgeschenke großmächtiger Könige waren daselbst dargebracht; gefeiert ward er als achtes Mysterium,\footnote{\begin{armenian}պաշտօն\end{armenian} Cultus Mysterium.} genannt das des drachenwürgenden Vahagn,\footnote{\begin{armenian}վիշապաքաղն Վահագնի\end{armenian}, der Grieche: \begin{greek}εὐφημοτάτου δρακοντοπνίκτου Ἡρακλέους\end{greek}.} die Opferstätten der Könige Großarmeniens auf dem Gipfel des Berges K͑ark͑ē oberhalb des Euphratstromes, welcher dem großen Berge Tauros gegenüberliegt und nach der Opfermenge der Stätten Yaštišat genannt wurde. Denn damals standen noch wohlgebaut daselbst 3 Altäre. Der erste Tempel (war) der Vahēvahische. Der zweite der der Goldmutter, der goldgeborenen Göttin, und auch der Altar hieß danach Goldkorn der Göttin Goldmutter. Der dritte Tempel hieß der der Göttin Astłik, das Gemach des Vahagn genannt. Diese ist nach dem Griechischen Ap̔rodites selbst.“\footnote{Der Grieche hat: § 140. \begin{greek}καὶ ὁ τρίτος βωμὸς ἀστέρος θεῶν καὶ τῶν Ἡρακλέους ἐλέγετο κληθείς κατὰ δὲ τοὺς Ἕλληνας Ἀφροδίτης\end{greek}. Nach dem armenischen Text (\begin{armenian}սենեակ Վահագնի\end{armenian}) ist beim Griechen \begin{greek}κοιτὼν Ἡρακλέους\end{greek} herzustellen. Die Worte \begin{armenian}ըստ յունական\end{armenian} müssen nach Anleitung der griechischen Übersetzung zu dem Folgenden gezogen werden.}

Man wird nicht behaupten, dass dieser Bericht sich durch besondere Klarheit auszeichne. Zuerst wird berichtet von einem Tempel (\begin{armenian}մեհեան\end{armenian}, mehean) dann werden drei Altäre (bagin \begin{armenian}երեք բագինքն\end{armenian}) erwähnt; hierauf bei der Einzelbeschreibung werden drei Tempel (Mehean) aufgezählt und bei dem zweiten ausdrücklich der Altar (bagin) vom Tempel unterschieden. Der Grieche übersetzt: \begin{greek}τοῦτο τὸ ἱερὸν ἔτι περιέστηκεν τρεῖς βωμοὺς ἐν ἑαυτῷ ἔχον\end{greek}.\footnote{Agathangelos neu herausg. von Lagarde (Abh. d. hist. phil. Cl. d. G. d. W. z. Göttingen 1889 35. 1) S. 71, 50. Ganz ebenso sind 71, 46: \begin{greek}προσηγορεύετο δὲ ὁ βωμὸς ὀγδόου σεβάσματος τοῦ εὐφημοτάτου δρακοντοπνίκτου Ἡρακλέους\end{greek} die Worte \begin{greek}ὁ βωμός\end{greek} erläuternder Zusatz, wie hier \begin{greek}τοῦτο τὸ ἱερόν\end{greek}. In der Übersetzung der Worte mehean und bagin ist der Grieche übrigens keineswegs konsequent. \begin{armenian}մեհեան\end{armenian} (mehean) gibt er durch \begin{greek}ἱερόν\end{greek} wieder.\hspace*{5mm}\begin{table}[H]
    \centering
    \tiny
    \begin{tabular}{p{45mm}|p{45mm}}
    \hline
        S. 452: \begin{armenian}յԱնահտական  ՚ի մեհեանն\end{armenian} & 65, 27: \begin{greek}εἰς τὸ τῆς Ἀρτέμιδος ἱερόν\end{greek}.   \\ \hline
        S. 453: \begin{armenian}՚ի դուռն մեհենին\end{armenian} & 65, 31: \begin{greek}εἰς τὴν θύραν τοῦ ἱεροῦ\end{greek}.   \\ \hline
        S. 455: \begin{armenian}մեհեան անուանեալ սպիտակափառ\end{armenian} & 66, 70: \begin{greek}ἱερὸν λεγόμενον λευκοδόξων δαιμόνων\end{greek}.   \\ \hline
        S. 457: \begin{armenian}՚ի մեծ և ՚ի բուն մեհենեացն Հայոց թագաւորացն\end{armenian} & 67, 86: \begin{greek}ἐν τοῖς μεγάλοις ἱεροῖς τῆς Ἀρμενίας\end{greek}.   \\ \hline
        S. 457: \begin{armenian}զգանձս երկուցուն մեհենացն\end{armenian} & 67, 95: \begin{greek}τοὺς δὲ θησαυροὺς ἀμφοτέρων τῶν ἱερῶν\end{greek}.   \\ \hline
        S. 459: \begin{armenian}՚ի Մրհական մեհեանն\end{armenian} & 68, 16: \begin{greek}ἑν τῷ ἱερῷ Ἡφαίστου\end{greek}. \\ \hline
    \end{tabular}
\end{table}
\hspace*{5mm}Durch \begin{greek}βωμός\end{greek} gibt er es wieder:  
\begin{table}[H]
    \centering
    \tiny
    \begin{tabular}{p{45mm}|p{45mm}}
    \hline
        S. 457: \begin{armenian}՚ի տեղիս պաշտամանացն յԱնահտական մեհենացն\end{armenian} & 67, 87: \begin{greek}ἐν τοῖς τόποις τῶν σεβασμάτων, ὅπου ἦν ὁ βωμὸς τῆς Ἀρτέμιδος\end{greek}.   \\ \hline
        S. 457: \begin{armenian}զՆանէական մեհեանն\end{armenian} & 67, 94: \begin{greek}τὸν τῆς Ἀθηνᾶς βωμόν\end{greek}.   \\ \hline
        S. 469: \begin{armenian}Վահէվահեան մեհեանն\end{armenian} & 71, 44: \begin{greek}Οὐαυήϊος βωμός\end{greek}.   \\ \hline
        S. 469: \begin{armenian}մեհեանն մեծագանձ\end{armenian} & 71, 44: \begin{greek}βωμὸς πλουσιώτατος\end{greek}.   \\ \hline
        S. 469: \begin{armenian}և երրորդ մեհեանն\end{armenian} & 71, 52: \begin{greek}καὶ ὁ τρίτος βωμός\end{greek}.   \\ \hline
        S. 470: \begin{armenian}յանդիման մեհենացն\end{armenian} & 71, 58: \begin{greek}κατ' ἔναντι τῶν βωμῶν\end{greek}.   \\ \hline
        S. 470: \begin{armenian}՚ի... տեղիս մեհենիցն\end{armenian} & 71, 59: \begin{greek}ἔνθα οἱ βωμοί\end{greek}. \\ \hline
    \end{tabular}
\end{table}
\hspace*{5mm}\begin{armenian}Բագին\end{armenian} (bagin) übersetzt er in der Regel durch \begin{greek}βωμός\end{greek}.  
\begin{table}[H]
    \centering
    \tiny
    \begin{tabular}{p{45mm}|p{45mm}}
    \hline
        S. 452: \begin{armenian}աւերել անդ զբագինսն Անահտական դիցն\end{armenian} & 65, 19: \begin{greek}καταστρέψαι τὸν ἐκεῖσε βωμὸν τῆς Ἀρτέμιδος\end{greek}.   \\ \hline
        S. 457: \begin{armenian}զբագինսն ևս դիցն Արամազդայ\end{armenian} & 67, 83: \begin{greek}τὸν βωμὸν Κρόνου\end{greek}.   \\ \hline
        S. 469: \begin{armenian}երեք բագինքն\end{armenian} & 71, 50: \begin{greek}τρεῖς βωμούς\end{greek}.   \\ \hline
        S. 470: \begin{armenian}՚ի յայս բագինս մնացեալս\end{armenian} & 71, 55: \begin{greek}ἐν τοῖς καταλειφθεῖσιν βωμοῖς τούτοις\end{greek}.   \\ \hline
        S. 471: \begin{armenian}զշինուածս բագնացն\end{armenian} & 72, 68: \begin{greek}τοὺς βωμούς\end{greek}. \\ \hline
    \end{tabular}
\end{table}
\hspace*{5mm}Aber ebenso findet sich \begin{greek}ἱερόν\end{greek}:  
\begin{table}[H]
    \centering
    \tiny
    \begin{tabular}{p{45mm}|p{45mm}}
    \hline
        S. 455: \begin{armenian}զի և անդ զանուանելոցն զսռւտ աստուածոցն զբագինսն կործանեսցեն\end{armenian} & 66, 68: \begin{greek}ἵνα κἀκεῖ τῶν ψευδωνύμων θεῶν τὰ ἱερὰ καταστρέψωσιν\end{greek}.   \\ \hline
        S. 471: \begin{armenian}և ոչ կարացին զդուրս բագնացն գտանել\end{armenian} & 72, 69: \begin{greek}οὐκ ἴσχυσαν ἐξευρεῖν τὰς θύρας τοῦ ἱεροῦ\end{greek}. \\ \hline
    \end{tabular}
\end{table}\hspace*{5mm}Für \begin{armenian}մեհեան\end{armenian} stehen auch durch den biblischen Sprachgebrauch sowohl die Bedeutungen Tempel, als Götterbild und Altar fest; (s. Hübschmann, arm. Gramm. 1. S. 194 s. v. \begin{armenian}մեհեան\end{armenian}) aber auch \begin{armenian}բագին\end{armenian} heißt nicht nur schlechtweg Altar, wie namentlich das an letzter Stelle erwähnte Zitat erweist. N. Tommaseo übersetzt mehrfach (z. B. S. 108 = 452; 151 = 457; 162 = 469 \begin{armenian}և բագինն իսկ յայս անուանեալ Ոսկիհատ Ոսկիամօր դիցն\end{armenian} u. s. f.): 'simulacro,' wohl mit Recht; denn es scheint das Wort den Altar mit dem zugehörigen Kultbild zu bezeichnen. Indessen an unsrer Stelle, wo der Geschichtsschreiber so scharf zwischen \begin{armenian}մեհեան\end{armenian} und \begin{armenian}բագին\end{armenian} unterscheidet, muss notwendig den beiden Worten eine verschiedene Bedeutung innewohnen.} Indessen solche erklärende Zusätze flicht er mehrfach seiner Übersetzung ein. Die Entscheidung gibt hier Faustus 3. 14 S. 37: \begin{armenian}մեծին Գրիգռրի յաւուրս յորում կործանեաւց զբագինսն մեհենիցն Հէրակլեայ՝ այս ինքն Վահագնի, որումտեղոյ Աշտիշատն կարդացէալ\end{armenian}..., des großen Grigor, als er die Altäre der Tempel des Herakles — das ist des Vahagn — zerstörte an dem Orte, Aštišat genannt.“ Hier haben wir es deutlich mit einer Mehrheit nicht nur von Altären, sondern auch von Tempeln zu tun. Möglicherweise lassen sich diese Widersprüche vereinigen durch die Annahme, dass das Tempelgebäude in drei Zellen zerfiel; der Mittelbau war das Heiligtum des Vahagn, die beiden Seitenkapellen waren dem Dienst der beiden Göttinnen gewidmet. Damit würde es gut stimmen, dass die Cella der Astłik als Schlafgemach oder Zimmer des Vahagn bezeichnet wird.

Über Vahagn fließen die Quellen etwas reichlicher, weil neben Agathangelos auch Moses von Chorēn wertvolle Mittheilungen beisteuert. Moses hat den Gott euhemerisiert; er ist ein Sohn des von ihm konstruierten ersten Tigran und ein Bruder von Bab und Tiran. Diese Genealogie ist natürlich wertlos, umso wichtiger das bekannte kleine Fragment eines auf Vahagn bezüglichen Liedes: Moses Chor. 1. 31: ...Vahagn, von dem die Fabeln\footnote{\begin{armenian}առասպելք\end{armenian}.} erzählen:
\begin{quotation}
\small
Es kreiste der Himmel, [es kreiste] die Erde,

Es kreiste auch das purpurne Meer.

Geburtsschmerzen hatte im Meer das blutrote Schilfrohr.\footnote{\begin{armenian}եղեցնիկ\end{armenian} eigentlich cannula kleines Rohr.}

Durch des Rohres Schaft kam Rauch heraus;

Durch des Rohres Schaft kam Flamme heraus.

Und aus der Flamme ein Knäblein sprang.

Feuer hatt' es als Haar,

Feuer trug es als Bart.

Und seine Äuglein waren Sonnen.\footnote{Die Übersetzung nach Vetter.}
\end{quotation}
\paragraph{}
Dies sangen einige zum Bambirn; wir haben es mit unseren Ohren gehört. Nach diesem erzählten sie auch im Liede, dass er mit Drachen kämpfte und siegte. Sehr ähnliches den Heldentaten des Herakles sangen sie über ihn. Ja sie erzählten, dass er Gott geworden sei. Und im Lande der Virk͑, (Iberer), das Maaß seiner Gestalt (= Bildsäule) errichtend, verehrten sie dasselbe durch Opfer. Von ihm stammen (wörtl. sind) die Häuser der Vahnunier.“ Vahagn ist vor allem der Schlangenwürger. \begin{armenian}Անոյշ մայրն վիշապաց\end{armenian} „Anoyš die Mutter der Drachen“\footnote{Mos. Chor. 1. 31.} wird wohl ursprünglich seinem Mythus angehört haben. In den epischen Liedern von Gołt͑n ist sie bereits euhemerisiert und zu der von Tigran in Gefangenschaft geschleppten Mederkönigin degradiert; aber auch ihr Gatte Ašdahak ist aus der Gewitterschlange in den Mederkönig Astyages umgewandelt worden. In dem wichtigen achten Kapitel von Moses zweitem Buche, welches die ganze von den Aršakunierkönigen eingerichtete Adelsordnung aufzählt, werden auch die \begin{armenian}որդիք Վահագնի\end{armenian} die Söhne Vahagns oder \begin{armenian}Վահնունիք\end{armenian} Vahnunier eingereiht. Sie erhalten das Priestertum (\begin{armenian}Քրմութիւն\end{armenian}) des Vahagn und eine Satrapie (\begin{armenian}Նախարարութիւն\end{armenian}) ersten Ranges.

Mit welcher Zähigkeit das armenische Volk an Vahagn hing, zeigt des Moses gewiss glaubwürdiger Bericht, dass noch zu seiner Zeit d. h. im tiefchristlichen siebenten oder achten Jahrhundert das Lied von des Gottes Geburt gesungen wurde. Es ist bedauerlich, dass der Geschichtsschreiber uns nur dieses eine theogonische Bruchstück im Originaltext mitgeteilt und das Lied von den Kämpfen mit den Drachen, wovon der Gott den Beinamen Višapak͑al trägt, nur kurz und beiläufig berührt hat Der in dem Geburtsmythus noch deutlich erkennbare physische Charakter des Gottes ist nun späterhin stark zurückgedrängt und verblasst. Vahagn erscheint als Jagd- und Siegesgott. Bereits von Gutschmid\footnote{Kl. Schr. 3. S. 414.} hat die merkwürdigen Züge aus der Legende des h. Athenogenes, des Rechtsnachfolgers des Vahagn, zusammengestellt, welche ihn deutlich als Schutzherrn der Tiere des Waldes und der Jagd charakterisieren. Die anmutige Legende von der Hindin, welche den Heiligen begleitet und welche freiwillig an seinem Gedächtnisstage ein Hirschkalb als Opfer darbringt, ist offenbar ein alter Göttermythus in christlichem Gewande.

Hier ist wohl auch Gott \begin{armenian}Տիւր\end{armenian} Tiur einzureihen, den man mit Tīr, dem Planeten Merkur, bei den Persern zusammengestellt hat.\footnote{Vgl. jedoch Hübschmann, arm. Gramm. 1. S. 89 No. 1, der den Gott nur zweifelnd in Parallele zu Tīr stellt; ebenso Lagarde, arm. Stud. S. 152. Dagegen die Berichtigung Agathangelos S. 139 und Justi, Iranisches Namenbuch S. 325.} Über ihn besitzen wir den ebenso wichtigen, als dunkeln\footnote{„Eine im griechischen wie im armenischen gleich sehr verderbte Stelle.“ Lagarde, ges. Abh. S. 294.} Bericht des Agathangelos, den ich hier beisetze S. 452: \begin{armenian}Ապա ինքն իսկ թագաւորն խաղայր ամենայն զօրօքն հանդերձ ՚ի Վաղարշապատ քաղաքէ՝ երթեալ յԱրտաշատ քաղաք աւերել անդ զբագինսն Անահտական դիցն և որ յԵրազամոյն տեղին անուանեալ կարդայր։ նախ դիպեալ ՚ի Ճանապարհի երազացոյց երազնդհան պաշտման Տիւր դից դպրի գիտութիւն (գիտութեան?) քրմացն՝ անուանեալ դիւան դրչի Որմզդի ուսման Ճարտարութեան մեհեան\end{armenian}. „Da brach der König selbst mit samt dem ganzen Heere aus der Stadt Vałaršapat auf und zog nach der Stadt Artašat, um dort zu zerstören die Altäre der anahitischen Gottheit. Und an dem Orte, welcher Erazamoyn mit Namen genannt wurde, traf er zuerst an der (öffentlichen) Straße auf die Träume weisende, Träume deutende Kultusstätte des Gottes Tiur des Schreibers der Wissenschaft der Priester, genannt das Archiv des Sekretärs des Ormizd, der Tempel der Schule der Beredsamkeit.“ Der Grieche fand die Worte schon in übler Verfassung, wie seine Übersetzung weist § 128: \begin{greek}ἐπορεύθη μετὰ τοῦ στρατεύματος ἀπὸ Οὐαλεροκτίστης τῆς πόλεως εἰς τὴν Ἀρταξερξοκτίστην, ὥστε καταστρέψαι τὸν ἐκεῖσε βωμὸν τῆς Ἀρτέμιδος ἐν τοῖς λεγομένοις τόποις ὀνειρομούσοις. ἀπιοῦσι δὲ αὐτοῖς ὑπήντησε τόπος προσαγορευόμενος ὀνειροπόλων, ὀνειροδεικτῶν σέβασμα δαίμονος Γραμματέως καὶ γνώσεις ἱερέων. Τρίδις ἐκέκλητο βωμὸς Ἀπόλλωνος · χαρτουλαρίων Διὸς διδαχῆς εὐμαθήτου\end{greek} (Codex \begin{greek}εὐμάθητος\end{greek}) \begin{greek}ἱερέων\end{greek}. Dieser Text, der mehrfach von dem armenischen abweicht, hat aber schon den Satz \begin{armenian}և որ — կարդայր\end{armenian} unmittelbar hinter \begin{armenian}Անահտական դիցն\end{armenian} gelesen; das hat den Übersetzer veranlasst die Worte mit dem Vorhergehenden zu verbinden. Allein das erregt eine geographische Verwirrung. P. Tommaseo S. 147 der italienischen Übersetzung bemerkt ganz richtig: „Questo tempio d'Apollo, eretto da Ardasese 2., era fuori della città d'Ardassad, presse la via che conduceva a Valarsabad; l'altro d'Anaite era dentro la città stessa.“ Das ergibt sich aus dem Berichte des Moses 2. 49, wo er den Bau des Artemistempels in Artašat durch König Artašēs erwähnt und hinzufügt: „Aber die Bildsäule des Apollon richtet er außerhalb der Stadt auf nahe der (öffentlichen) Straße.“ Genauso bei Agathangelos. Der König verlässt seine Residenz Vałaršapat, um nach Artašat zu gelangen. Unterwegs trifft er auf das Tiurorakel in Erazamoyn; dieser Ort liegt also in der Mitte zwischen Kainepolis und Artaxata. Das zeigt auch der Fortgang des Berichts, worin erzählt wird, dass die Vertheidiger des Orakels geschlagen sich nach dem Tempel der Anahit, also nach Artaxata, zurückzogen, um dort im Verein mit den Priestern der Göttin den Kampf gegen das christliche Heer fortzusetzen. Nach alle dem leidet es keinen Zweifel, dass die Worte \begin{armenian}և որ — կարդայր\end{armenian} mit dem Nachfolgenden zu verbinden sind, wie auch der italienische Übersetzer tut, während Langlois sie unrichtig den vorangehenden Worten anschließt. Wahrscheinlich sind Übrigens diese Worte von ihrer ursprünglichen Stelle durch Versehen verschoben. Denn der Text in der jetzigen Gestalt zeigt eine unerträgliche Härte. \begin{greek}Τρίδις\end{greek} des Griechen ist, wie Lagarde (Agathangelos S. 139) gesehen, eine einfache Transkription von \begin{armenian}Տրի դից\end{armenian}; der Grieche las also nicht \begin{armenian}Տիւր դից\end{armenian}, sondern die Genetivform. Die genauen und höchst beachtenswerten Angaben über die unter dem Schutze des Orakelgottes stehende Weisheitsschule scheinen auf einen in diese Geheimnisse eingeweihten Verfasser, vielleicht einen ehemaligen K͑urm, zu deuten. Bekanntlich ist das Leben des h. Gregor von einem wohlunterrichteten Zeitgenossen geschrieben. Beachtenswert ist auch, dass statt der sonst in Armenien gebräuchlichen Form Aramazd hier uns bereits das sāsānidische\footnote{Lagarde, ges. Abh. S. 292. „\begin{armenian}Որմիզդ\end{armenian} gehört den in das christliche Armenien Einfälle machenden Sāsāniden.“ Agathangelos S. 139.} Ormizd \begin{armenian}գրիչ Որմզդի\end{armenian} „Sekretär des Ormizd“ begegnet. Man möchte daraus schließen, dass die mit dem Orakel von Erazamoyn verbundene Theologenschule ein Institut recht jungen Datums sei, organisiert vielleicht erst während der sāsānidischen Okkupation im 3. Jahrhundert.

Aufs innigste ist endlich mit dem religiösen Volksbewusstsein der Armenier die aus Iran entlehnte weibliche Gottheit Anāhita, armenisch \begin{armenian}Անահիտ տիկին\end{armenian} Anahit tikin \begin{greek}Ἄρτεμις δέσποινα\end{greek} verknüpft. Natürlich hat sich diese Erörterung auf die spezifisch armenische Ausgestaltung dieses Götterwesens zu beschränken. Mit Recht ist hervorgehoben worden, dass das ursprüngliche Wesen dieser Göttin unter dem Einflüsse babylonischer Anschauungen in den Westlandschaften eine völlige Umbildung erfahren hat,\footnote{Spiegel: Erânische Altertumskunde 2. S. 56 ff. E. Meyer, Gesch. d. Altertums 1. § 450. Roschers Lexikon d. gr. u. r. Mythol. Sp. 332.} und gerade die in Armenien verehrte Göttin zeigt vielfache Beimischung fremder Elemente. Ihre Hauptheiligtümer befinden sich in Akilisene-Ekełeac̣, einer ehemals zu dem von Syrern bewohnten Kataonien gehörigen Provinz und in Aštišat, also in der gleichfalls ursprünglich syrischen Provinz Taraunitis (Tarōn-Duruperan). Außerdem ist ihr Kult in der alten Reichshauptstadt Artašat nachweisbar; das ist wohl erklärlich, da die Göttin in die Dreizahl der obersten Schutzgottheiten — gleichsam der Penates publici Armeniens — Aramazd, Anahit, Vahagn Aufnahme gefunden hatte, wie dies das Dekret des Königs Tiridates zeigt. Für den Kult der Göttin in Armenien liegen recht alte Zeugnisse vor. Cicero\footnote{De imperio Pompei 23.} berichtet, dass, als Lucullus in Armenien eindrang, eine ungeheure Erregung religiöser Art die dortigen Völkerschaften ergriff: „Plures etiam gentes contra imperatorem nostrum concitatae sunt. Erat enim metus iniectus iis nationibus quas nunquam populus Romanus neque lacessendas bello, neque temptandas putavit. erat etiam alia gravis atque vehemens opinio quae per animos gentium barbararum pervaserat, fani locupletissimi et religiosissimi diripiendi causa in eas oras nostrum exercitum esse adductum.“ Bei diesem reichen, von den umliegenden Nationen verehrten Sanktuarium denkt Mommsen an den Tempel der persischen Nanaea oder Artemis in Elymais „das gefeiertste und das reichste Heiligtum der ganzen Euphratlandschaft.“\footnote{Mommsen, R. G. 3. 7, S. 72.} Indessen die Schätze der Elymäischen Heiligtümer — angeblich 10000 Talente — hatten bereits die Parther geplündert.\footnote{Strabo 16. p. 744 f. Es geschah das wohl auf dem Feldzuge, in welchem Arsakes Mithradates 1. ( 138) den König der Elymäer unterwarf. Justin. 41., 6. 8 vgl. Gutschmid: Geschichte Irans S. 53.} Mehr ins Gewicht fällt die große geographische Entfernung, welche uns nicht an ein susianisches Heiligtum denken lässt. Am wahrscheinlichsten scheint es mir darunter das hochberühmte Heiligtum der Anahit in Erēz (\begin{greek}Ἔριζα\end{greek}) zu verstehen. Der Tempel von Akilisene ist auch in späterer Zeit das Ziel der römischen Raubzüge. Plinius berichtet, dass die goldene Statue desselben im Partherkrieg des Antonius von dessen Soldaten zerschlagen und eingeschmolzen ward. Dabei redet er von der Göttin in einer Weise, dass wir in ihrem Sanktuarium unschwer Ciceros fanum locupletissimum et religiosissimum wiedererkennen. Plinius N. H. 33. 82: Aurea statua prima omnium nulla inanitate et antequam ex aere aliqua illo modo fieret, quam vocant holosphyraton, in templo Anaetidis posita dicitur, quo situ terrarum nomen hoc signavimus, numine gentibus illis sacratissimo. direpta ea est Antonii Parthicis rebus, scitumque narratur veteranorum unius Bononiae hospitali divi Augusti cena, cum interrogaretur, essetne verum eum qui primus violasset id numen oculis membrisque captum exspirasse; respondit enim tum maxime Augustum e crure eius cenare seque illum esse totumque sibi censum ex ea rapina. Dass eine solche Legende von dem angeblichen Strafgericht der Göttin entstehen und bis nach Rom Verbreitung finden konnte, beweist die hohe Verehrung, in welcher die dortige Anahit stand. Über die Lage hat sich Plinius 5. 83 geäußert, auf welche Stelle er hier anspielt: fluit (sc. Euphrates) Derzenen primum, mox Anaeticam Armeniae regiones a Cappadocia excludens. Es ergibt dies deutlich die Identifikation der Anaetica mit Akilisene, welche auch bei Cassius Dio\footnote{36, 48: \begin{greek}τὴν Ἀναῖτιν χώραν τῆς τε Ἀρμενίας οὖσαν καὶ θεῷ τινι ἐπωνύμῳ ἀνακειμένην\end{greek} vgl. 53.} \begin{greek}ἡ Ἀναῖτις χώρα\end{greek} heißt. Auch Strabo, in kleinasiatischen Dingen vorzüglich unterrichtet, erwähnt das akilisenische Heiligtum als Zentralstätte des Anaitiskultes; die mit dem Kulte verbundene Hierodulie und die Prostitution der Töchter der Vornehmen zeigen, dass entschieden nichtiranische Elemente sich in Armenien mit dem Anahitkultus verbunden haben. Strabo 11. 532 C: \begin{greek}Ἅπαντα μὲν οὖν τὰ τῶν Περσῶν ἱερὰ καὶ Μῆδοι καὶ Ἀρμένιοι τετιμήκασι, τὰ δὲ τῆς Ἀναΐτιδος διαφερόντως Ἀρμένιοι, ἔν τε ἄλλοις ἱδρυσάμενοι τόποις, καὶ δὴ καὶ ἐν Ἀκιλισηνῇ. ἀνατιθέασι δ' ἐνταῦθα δούλους καὶ δούλας. καὶ τοῦτο μὲν οὐ θαυμαστόν, ἀλλὰ καὶ θυγατέρας οἱ ἐπιφανέστατοι τοῦ ἔθνους ἀνιεροῦσι παρθένους, αἷς νόμος ἐστὶ καταπορνευθείσας πολὺν χρόνον παρὰ τῇ θεῷ μετὰ ταῦτα δίδοσθαι πρὸς γάμον, οὐκ ἀπαξιοῦντος τῇ τοιαύτῃ συνοικεῖν οὐδενός\end{greek}.“ Die offenbar recht junge Legende von der Göttin gibt Prokop. Die gefeiertsten Heiligtümer der Anaitis und der Ma rühmten sich, von Orest und Iphigenia gestiftet zu sein und behaupten, dass das Gnadenbild der Tauropolos zu ihnen verbracht sei, so vor allem das kappadokische Komana am Saros (Strabo 12. 535 C) und Hierokaisareia in Lydien. (Pausan. 3. 16, 8; 7. 6, 6.) Die armenische Legende\footnote{Über dieselbe vgl. auch die Ausführungen von G. Hoffmann: Auszüge aus den syr. Akten pers. Märtyrer S. 135.} geht nun noch einen Schritt weiter und überbietet die der andren Völker. Kühn versetzt sie „\begin{greek}τὸ ἐν Ταύροις Ἀρτέμιδος ἱερόν\end{greek}“ aus der taurischen Halbinsel nach Armenien. Procop de bello Persico 1. 17 p. 83: \begin{greek}ἐνθένδε τε ὁ ποταμὸς πρόεισιν εἰς τὴν Ἐκελεσηνὴν καλουμένην χώραν, οὗ δὴ τὸ ἐν Ταύροις τῆς Ἀρτέμιδος ἱερὸν ἦν, ἔνθεν λέγουσι τὴν Ἀγαμέμνονος Ἰφιγένειαν ξύν τε Ὀρέστῃ καὶ Πυλάδῃ φυγεῖν τὸ τῆς Ἀρτέμιδος ἄγαλμα φέρουσαν. ὁ γὰρ ἄλλος νεὼς ὃς δὴ καὶ ἐς ἐμέ ἐστιν ἐν πόλει Κομάνῃ, οὐχ ὁ ἐν Ταύροις ἐστίν\end{greek}. Prokop führt dann des weitern aus, dass Orestes auf der Flucht mit der Schwester zuerst das pontische Komana und dann das kappadokische gegründet habe. Das Orakel hatte ihm das Ende seiner Leiden prophezeit an einer Stätte, welche dem taurischen Mutterheiligtum ähnlichsähe. Das trifft nun auch nach Prokops Urteil auf Komana am Saros zu a. a. O. S. 84: \begin{greek}ὅνπερ (χῶρον) καὶ ἐγὼ πολλάκις ἰδὼν ἠγάσθην τε ὑπερφυῶς καί μοι ἐδόκουν ἐν Ταύροις\end{greek} (= in Eriza) \begin{greek}εἶναι. τό τε γὰρ ὄρος τοῦτο ἐκείνῳ ἀτεχνῶς ἔοικεν, ἐπεὶ κἀνταῦθα ὁ Ταῦρός ἐστι, καὶ ὁ ποταμὸς Σάρος τῷ Εὐφράτῃ εἰκάζεται\end{greek}. Man könnte danach annehmen, dass wir es lediglich mit einer Konstruktion Prokops zu tun haben. Allein dem widerspricht die Parallelstelle,\footnote{De bello Gothico 4. 5 p. 480: \begin{greek}καίτοι Ἀρμένιοι ἐν τῇ παρ' αὐτοῖς Ἀκιλισηνῇ καλουμένῃ χώρᾳ τὸν νεὼν τοῦτον γεγονέναι φασὶ καὶ Σκύθας τηνικάδε ξύμπαντας καλεῖσθαι τοὺς ἐνταῦθα ἀνθρώπους\end{greek}.} wo er ausdrücklich die Armenier als Gewährsmänner für das Dogma von der Urheimat der taurischen Artemis in Akilisene anführt. Wie wenig sich diese neue Legende um die älteren Versionen kümmert, zeigt auch der Umstand, dass das pontische Komana hier zuerst gegründet wird, während Strabo (12. 557 C.) es ausdrücklich zum Filial des kappadokischen macht. Die ganze Konstruktion ist als gelehrte Kombination eines hellenistisch gebildeten Armeniers immerhin bemerkenswert. Wenn man eine Vermutung wagen darf, könnte vielleicht angenommen werden, dass Olympios von Ani in seiner Tempelgeschichte Eriza zum Mutterheiligtum der beiden Komana gemacht habe.

Ausführlich und authentisch sind wir über den Kult des 3. Jahrhunderts durch Agathangelos unterrichtet. Die Frömmigkeit der Umwohner und Waller hat das von den Römern einst zerschlagene Götterbild wohl bald wieder aufgerichtet. Wenigstens findet dreihundert Jahre später der h. Gregor aufs neue Gelegenheit, eine Goldstatue der Anahit zu zerstören. Agathang. S. 457: „Nach diesem zog er nach der benachbarten Provinz Ekełeac̣. Dort erschienen die Dämonen in den großen und ursprünglichen Heiligtümern der armenischen Könige, an den Stätten der Verehrung der anahitischen Idole\footnote{\begin{armenian}՚ի տեղիս պաշտամանացն յԱնատական մեհենացն\end{armenian}. Obwohl hier der Plural gebraucht wird, ist nachher nur von einem Götterbilde die Rede.} in dem Flecken Erēz; nach Art eines schildbewaffneten Heeres sammelten sich die Dämonen und kämpften, und von ihrem lauten Getöse machten sie die Berge wiederhallen. Diese begaben sich auf die Flucht und gleichzeitig mit ihrem Entweichen wurden die sehr hohen Wälle zerstört und dem Erdboden gleich gemacht. Herausstürmend, eindringend zermalmten das muntere (?)\footnote{\begin{armenian}զգաոտացեալ զօրօքն\end{armenian}.} Heer und der heilige Grigor mitsamt dem Könige das goldene Bildnis der Göttin Anahit. Die ganze Stätte zerstörten, vernichteten sie. Das Gold und Silber schleppten sie als Beute fort.“

Ein zweites goldnes Bildnis ist das in Tarōn befindliche, wo sie mit Vahagn und Astłik die Trias von Aštišat bildet.\footnote{Die Beschreibung des Bildes S. 104 und 105.} Die Fülle des Ausdrucks\footnote{S. 469: \begin{armenian}մեհեանն Ոոկեմօր ոոկեծին դից und բագինն Ոսկիհատ Ոսկիամօր դիցն\end{armenian} und beim Griechen: (\begin{greek}βωμὸς\end{greek}) \begin{greek}χρυσῆς μητρὸς χρυσογεννήτων θεῶν χρυσόκοκκος: χρυσῆς μητρος τῶν δαιμόνων\end{greek}.} lässt uns vermuten, dass wir es hier mit einem Bruchstück eines alten Tempelhymnus auf die Goldmutter zu tun haben. Ein drittes hochwichtiges Heiligtum der Anahit befand sich nach Agathangelos (S. 452 und 453) in der alten Königsstadt Artaxata. Indessen von dem dortigen Kulte berichtet unsre Quelle nichts eigentümliches. Immerhin muss auch dies ein sehr bedeutendes Heiligtum gewesen sein. Die Macht der dortigen Priesterschaft erweist nämlich ihr fanatischer Widerstand gegen das königliche Heer. Geschlagen floh das Priesterheer in die Tempelburg, und von den Zinnen des Heiligtums schossen die Verzweifelten „kraftlose Pfeile und Steine wie Hagel“ auf die Belagerer; wodurch die Neubekehrten anfangs sehr erschreckt werden. Allein die von Gregor begeisterten Truppen nehmen den Tempel mit stürmender Hand und demolieren ihn in üblicher Weise.

Eine letzte heilige Stätte wird von Faustus 5. 25 erwähnt als Wohnsitz des griechischen Anachoreten Epip͑an (Epiphanios): \begin{armenian}և նստէր սա ՚ի մեծի լերինն ՚ի տեղի դիցն ՝զոր կոչեն աթոր Նահատայ\end{armenian}. „Und er saß auf dem großen Berge an der Stätte der Götzen, welche sie Thron der Nahat nennen.“ In der Venetianerausgabe ist \begin{armenian}Անահտայ\end{armenian} geschrieben, obgleich alle von den Herausgebern benutzten Handschriften \begin{armenian}Նահատայ\end{armenian} lesen, entsprechend der persischen Nebenform Nāhēt.\footnote{Noeldeke Ṭabari S. 4 Nr. 2. Spiegel Er. Alt. 2. S. 54 Nr. 1.} Bei der Nachricht vom Tode des h. Nersēs begeben Epiphanios und der Anachoret Šałita sich sogleich nach Thil, der Begräbnisstätte der armenischen Hohepriester, in die Provinz Ekełeac̣. „Der Thron der Nahat“ wird demnach nicht zu ferne von dieser Gegend zu suchen sein.

Über den im 3. Jahrhundert gebräuchlichen Opferritus der Göttin gibt wiederum Agathangelos (S. 39) eine wichtige und authentische Nachricht. „Im ersten Jahre von Trdats Königsherrschaft über Großarmenien zogen sie nach der Provinz Ekełeac̣ in das Dorf Erēz zu dem Tempel der Anahit, um daselbst ein Opfer darzubringen. Und als sie das Werk der Unwürdigkeit vollbracht hatten, stiegen sie hinab, lagerten sich am Ufer des Stromes, welchen sie Gayl (Lykos) nennen. Als er daselbst in das Zelt getreten war und sich zum Mahle gesetzt hatte, und als sie vom Weine trunken wurden, gab der König dem Grigor Befehl, dass er Kränze und Büsche von Bäumen als Weihegabe darbringe auf dem Altar des Anahitischen Götterbildes.“ Auf Grigors Weigerung diese Kulthandlung zu vollziehen gerät der König in furchtbaren Zorn und droht ihm (S. 40) mit Gefängnis, Ketten und Tod, „wenn du dich nicht dazu verstehst den Göttern Verehrung zu bezeigen, besonders dieser großen Herrin Anahit, welche der Ruhm unsres Volkes und die Lebensgeberin ist, welche auch alle Könige verehren, zumal der Kaiser der Griechen, welche die Mutter aller Weisheit, die Wohltäterin der gesamten menschlichen Natur und die Tochter des großen und tapferen Aramazd ist.“ Diese überschwängliche Verehrung der Allmutter ist für das spätere Heidentum charakteristisch. Bemerkenswert ist, dass unblutige Opfer, Blumenkränze und Baumzweige, der Göttin dargebracht werden. Zu beachten ist auch, dass sich im Leben Gregors keine Spur mehr von der im Dienste der Göttin vollzogenen Prostitution findet, deren Strabo gedenkt. Was sonst durch zahlreiche Beispiele zu belegen ist, scheint demnach auch für Armenien zuzutreffen. Die Vorwürfe der monotheistischen Gottesverehrer, der alexandrinischen Juden und der christlichen Apologeten, waren nicht ohne Einwirkung auf den heidnischen Theil der Bevölkerung geblieben. Der Paganismus des 3. und 4. nachchristlichen Jahrhunderts ist entschieden sittlicher, als der der Vorzeit. Eine indirekte Stütze unsrer Ansicht gewähren auch die Worte des h. Gregor, worin er das Wesen der Göttin Anahit ganz in der euhemeristischen Ausdrucksweise der christlichen Apologeten definiert. Agathangelos S. 42: „Aber bezüglich derjenigen, welche Du die große Herrin Anahit nennst, so hat es vielleicht einmal vor alter Zeit solche Menschen gegeben. Denn durch götzendienerische Zauberei und Visionen der verschiedene Gestalten annehmenden Dämonen haben sie die Menschen, welche damals lebten, überredet, Tempel zu erbauen, Bildsäulen zu errichten und den Staub zu küssen. Aber diese haben keine Existenz. Weder etwas böses, noch etwas gutes können sie vollbringen u. s. f.“ Gregor verbreitet sich noch des weitern über die völlige Nichtigkeit der Heidengötter. Hätte er irgendwie die Unsittlichkeit des Anahitkultus rügen können, gewiss hätte er dem Könige gegenüber das Argument mit voller Wucht zur Geltung gebracht, dass jene zauberkundige Frau der Vorzeit die Menschen zu den schrecklichsten Greueln verführt hätte. Sein Schweigen ist doch bedeutsam; es scheint mir dafür zu sprechen, dass das 3. Jahrhundert zu Erēz keine Prostitution mehr im göttlichen Dienste kannte, Anahit ist entschieden die am meisten verehrte und am höchsten gehaltene Gottheit des alten Armeniens. Darum ist die Weigerung ihr zu dienen ein unsühnbares Verbrechen. So sagt König Trdat, Agath. S. 47: „Darum habe ich Mitleid mit Dir gehabt, als einem bewährten Diener, damit Du wieder in die Ordnung der Rechtmäßigkeit eintretest die Götter zu verehren. Aber Du beraubst sie ihrer Ehre, einen nichtigen Schöpfer rufst Du an. Und den, welcher in Wahrheit der Schöpfer ist, verhöhnst Du, und die große Anahit, durch welche das Land der Armenier lebt und Leben erhält, und mit ihr auch den großen und tapferen Aramazd, den Schöpfer Himmels und der Erden, und mit ihm hast Du auch die übrigen Götter stumm und sprachlos genannt.“

Mit Anahit ist der Kreis der iranischen Gottheiten abgeschlossen. Wenn wir die Eigenart dieser Götterkulte betrachten, zeigen sie eine bemerkenswerte Verschiedenheit gegenüber den andren Filialen iranischer Religionslehre, welche in Kleinasien, vorab in Kappadokien und Lydien, nachweisbar sind. Dort wird überall mit einem gewissen Selbstbewusstsein der persische Ursprung dieser Gottesdienste betont. In Zela im pontischen Kappadokien wird eine Trias \begin{greek}συμβώμων θεῶν\end{greek} verehrt, neben Anaitis zwei ausdrücklich \begin{greek}Περσικοὶ δαίμονες\end{greek} genannte Gottheiten.\footnote{Vgl. Strabo 15. 733 C. \begin{greek}πολλὰ δὲ καὶ ἐν τῇ Καππαδοκίᾳ τῶν Περσικῶν θεῶν ἱερά\end{greek}.} Die persischen Feldherrn, nach einer zweiten Version der Legende, Kyros selbst, gelten als Gründer des Heiligtums (Strabo 11. 512 f.). In Lydien, wo persische Kolonien bestanden, (Strabo 13. 629 C.) werden Heiligtümer der persischen Lyder (\begin{greek}Λυδοῖς ἐπίκλησιν Περσικοῖς\end{greek} Pausan. 5. 27, 5) in Hierokaisareia und in Hypaipa erwähnt. Die \begin{greek}Ἄρτεμις Ἀναιῖτις\end{greek} der Lyder (Pausan. 3. 16, 8) wird ausdrücklich als persische Artemis\footnote{Persica Diana Tac. Ann. 3 62. \begin{greek}ἱερὸν Περσικῆς Ἀρτέμιδος\end{greek} Paus. 7. 6, 6.} bezeichnet. Nichts ähnliches in Armenien. Ferner sind in Kappadokien die Magier äußerst zahlreich; die Feueraltäre (\begin{greek}πυραιθεῖα\end{greek}) und der spezifisch persische Magierritus werden uns hier von einem Augenzeugen als bestehend geschildert. (Strabo 15. 733 f.). Derselbe Ritus wird uns in genauer Beschreibung für Lydien bezeugt (Pausan. 5. 27, 5). Dagegen in Armenien finden wir weder Magier noch Feuerkultus. Mögen jene kappadokischen und lydischen Magier dem orthodoxen Mazdaismus ziemlich ferngestanden haben, das aus Iran entlehnte Gut in Armenien steht ihm jedenfalls noch ganz anders fremdartig gegenüber. Theils wird es national umgearbeitet, wie dies namentlich bei der Gestalt des Vahagn ersichtlich ist, teils erscheint es mit Elementen ganz anderen Ursprungs verbunden. Wir finden hier ein ähnliches Göttergemisch, wie es uns auf den Münzen der indoskythischen Fürsten entgegentritt. Wenn wir nun bedenken, dass gerade diese die wichtigsten gleichzeitigen Denkmäler für den Mazdaismus sind in der Periode zwischen dem Achämeniden- und dem Sāsānidenreiche, wenn wir ferner erwägen, dass alles iranische in Armenien spezifisch parthisch ist, wird man wohl am wahrscheinlichsten das Eindringen iranischer Göttergestalten nach Armenien der altern parthischen Epoche zuschreiben. Wenn auch unter den späteren Arsaciden das nationale und religiöse Element wieder einen Aufschwung nahm, die älteren waren schon durch ihren Philhellenismus tolerant und laue Zoroastrier, und die Beeinflussung von dieser Seite erklärt den ketzerischen Charakter der unter diesen Fürsten importierten armenischen Religion. So begreift es sich, dass die Sāsāniden, sobald sie in Armenien erobernd vordringen, sich gleich in religiös feindlichen Gegensatz zu der Nation setzen. Sie brechen die Götter- und Heroenbilder, und errichten zwangsweise auf dem Berge Bagavan einen Feuertempel. Man sieht, der Gegensatz zwischen den heidnischen Armeniern und den wahren Mazdaisten ist so groß als möglich.
\clearpage
\section{Gottheiten syrischen Ursprungs}
\paragraph{}
Neben den starken Eindrücken, welche die armenische Götterlehre von iranischer Seite empfangen hat, sind vor allem auch die syrischen Einwirkungen zu betrachten. In der Perserzeit war aramäisch die Sprache nicht nur in Syrien und dem oberen Mesopotamien, sondern hatte auch in Irak die alte Landessprache, das Assyrisch-Babylonische verdrängt. In Kleinasien haben, wie die Münzen erweisen, die persischen Satrapen dasselbe zur offiziellen Sprache erhoben. Einwirkungen von dieser Seite schreibt aber die einheimische armenische Überlieferung erst der hellenistischen Epoche zu, und es wird sich zeigen, dass vieles für diese Auffassung spricht. Erst die unmittelbaren Vorgänger Tigranes des Großen haben das vorher unbedeutende Armenien nach allen Seiten erweitert und von Kataonern (Akilisene) und Syrern (Taraunitis) bewohnte Landstriche dem Reiche nicht allein einverleibt, sondern auch armenisiert. Die Armenier haben es also verstanden diese fremden Bestandteile der Bevölkerung aufzusaugen und sich völlig zu eigen zu machen.\footnote{\begin{greek}συνηύξησαν ἐκ τῶν περικειμένων ἐθνῶν ἀποτεμόμενοι μέρη... Καταόνων δὲ Ἀκιλισηνὴν καὶ την περὶ τον Ἀντίταυρον, Σύρων δὲ Ταραυνῖτιν, ὥστε πάντας ὁμογλώττους εἶναι.\end{greek} Strabo 11. 528 C.} Es ist aber selbstverständlich, dass dieser Amalgamierungprozess nicht vor sich gehen konnte, ohne dass auch die Armenier dem kulturell höherstehenden Volke gegenüber mehrfach als die entlehnenden auftreten.

Zuerst ist hier der Gott Baršamin zu nennen. Über ihn berichtet Agathangelos S. 455: „Er zog aus, gelangte (nämlich der h. Gregor) nach der Provinz Daranałi, damit sie auch dort die Altäre (\begin{armenian}զբագինոն\end{armenian}) der sogenannten falschen Götter zerstörten, den Tempel des weißglänzend genannten Götzen Baršimnia (\begin{armenian}Բարշիմնիա\end{armenian}) welcher in dem Dorfe T͑ordan war. Zuerst zerstörten sie diesen und die Bildsäule desselben zermalmten sie.“ Damit vergleiche man die wegen der abweichenden Namensform wertvolle Version des Griechen § 131: \begin{greek}παραγενόμενος δὲ ἔφθασε τὴν τῶν Δαραναλιτῶν πατρίδα, ἵνα κἀκεῖ τῶν ψευδωνύμων θεῶν τὰ ἱερὰ καταστρέψωσιν. ἐλθόντες δὲ ἐν τῇ καλουμένῃ κώμῃ Θορδάν, ἐν ᾗ ὑπῆρχεν ἱερὸν λεγόμενον λευκοδόξων δαιμόνων, βωμὸς Βαρσαμήνης, πρῶτον τοῦτον καταστρέψαντες, τὴν εἰκόνα αὐτῆς συνέτριψαν\end{greek}. Der Grieche hält irrig die Gottheit für eine weibliche. Die Namensform des Griechen wird aber unterstützt durch den Bericht des Moses von Chorēn über die Einsetzung dieses Dienstes 2. 14: (Tigran) steigt nach Mesopotamien hinunter und findet daselbst die Bildsäule des Baršamin (\begin{armenian}Բարշամինայ\end{armenian}), welche aus Elfenbein, Krystall (\begin{armenian}Բիւրեղէ\end{armenian}) und Silber angefertigt war. Er gab den Befehl, sie fortzuschaffen und in dem Flecken T͑ordan aufzurichten.“ Der Bericht des Moses rechtfertigt das Epitheton \begin{armenian}սպիտակափառ\end{armenian} weissglänzend für den Gott. Seine Statue aus Silber und Elfenbein — aus Kristall waren wohl die eingesetzten Augen — musste einen hellen Glanz verbreiten. Eine interessante wohl auf ein altes Lied zurückgehende Sage von Baršamin hat Anania von Širak aufbewahrt, wo er von der Milchstraße \begin{armenian}յաղագս կաթին ծրոյ\end{armenian} handelt. Zuerst kommen mehrere der griechischen Mythologie entlehnte Erklärungen; dann folgt die altarmenische\footnote{\begin{armenian}Սնանիյի Շիրակունւոյ մնացորդք բանից\end{armenian}, des Anania von Širak Paralipomena seiner Werke herausg. v. K͑. Patkanean. Petersburg 1877 S. 48.}: „\begin{armenian}դարձեալ ոմանք յառաջնոյն Հայոց ասացին թե խիստ ձմերանի Վահագն նախնի\footnote{Eine Kopie nach den Handschriften von San Lazzaro verdanke ich der Gefälligkeit meines Freundes, des P. Basil Sargisean. Ein Codex hat \begin{armenian}նախնին\end{armenian}, der andre \begin{armenian}նախնի\end{armenian}.} Հայոց գողացաւ զյաւրն Բարշամայ\footnote{Ein Codex \begin{armenian}բաղամայ\end{armenian}. Patkanean: \begin{armenian}բաղաամայ\end{armenian}. Die Neben form \begin{armenian}բաղամ\end{armenian} ist wohl absichtliche Korruption = \begin{armenian}բաղաամ\end{armenian} \begin{greek}Βαλαάμ\end{greek} um den heidnischen Greuel zu verhüllen?} Ասորեստանեայց\footnote{\begin{armenian}Սսորւոց\end{armenian} Patkanean.} նախնոն, զոր և մեք սովորեցաք բնախօսութեամբ յարդագողի\footnote{\begin{armenian}յարդգողի\end{armenian} Patkanean.} հետ անուանէլ\end{armenian}. Wiederum einige von den alten Armeniern haben gesagt, dass im strengen Winter Vahagn der Häuptling der Armenier, Stroh stahl dem Baršam, dem Häuptling der Assyrer, weshalb auch wir uns gewöhnt haben in der Naturkunde sie die Spur des Strohdiebes zu nennen.“

Wichtig ist, dass hier Baršam (Bałam) ausdrücklich als Assyrer bezeichnet und als Feind dem Armenier Vahagn gegenübergestellt wird. Die Namensform \begin{armenian}Բարշամ\end{armenian} Baršam, findet sich wie bei Anania, auch in einer Erzählung, welche Moses dem Mar Abas entlehnt haben will; sie findet sich freilich nicht in dessen Auszug. Er berichtet (2. 14), dass Aram nach Beendigung seines Kampfes mit den Ostländern gegen Assyrien gezogen sei. In siegreichem Kampfe erschlägt er dort „einen Verderber seines Landes, Baršam mit Namen vom Geschlechte der Riesen.“ „Und diesen Baršam haben die Syrer wegen der vielen Taten seiner Tapferkeit zum Gott erhoben und lange Zeit verehrt.“ Also auch in diesem Bericht erscheint Baršam als Repräsentant der Syrer im Gegensatz zu (dem freilich keineswegs genuin armenischen) Aram, welcher ihm gegenüber das nationale Element vertritt.

Zur Erklärung des Namens Baršimnia Baršamin bemerkt Lagarde (Agathangelos S. 138): „Baršimnia sieht mir ganz so aus, als sei es aus $\svgAAAA$ (was ich nicht verstünde) verlesen worden: da indessen \begin{greek}Βαρσαμήνης\end{greek} abweicht, lasse ich den Namen bis auf weiteres unbesprochen. Hier mag G. Hoffmann für weitere Aufklärung sorgen.“ G. Hoffmann\footnote{Akten u. s. f. S. 136.} hat ihn übrigens bereits mit B'elšmīn kombiniert, dessen Kult er aus Syrien nach Armenien übertragen sein lässt. Der Himmelsherr ist im Phönikischen inschriftlich mehrfach nachweisbar\footnote{Für das Folgende E. Meyer Art. Ba'al bei Roscher myth. Lex. Sp. 2872.} und zwar in der Form Ba'alšamēm. Wenn er daher bei Sanchuniathon Beelsamen heißt\footnote{Euseb. praep. ev. 1., 10. 7.}: \begin{greek}Βεελσάμην ὅ ἐστι παρὰ Φοίνιξι κύριος οὐρανοῦ\end{greek}, so hat Philon bereits die aramäische Form Ba'alšamin angenommen. Gerade in Syrien wird der Himmelsbaal Be'elšamin viel verehrt. Schwierigkeit bereitet jedoch der Rhotacismus in der überlieferten armenischen Namensform. Er kann nicht anders, als mittelst Durchgang durch das Iranische erklärt werden. Man denke an Babiru Babilu; Naditabira = Nidintavbil u. s. f. Das ist aber nur denkbar in einer Periode, wo der iranische Einfluss auf die höheren (in Armenien allein in Betracht kommenden) Gesellschaftsschichten ein maßgebender geworden ist. Nach allem, was wir wissen, scheint das in Tigranes' Zeit bereits längst vollzogen zu sein. Damit stimmt nun aufs schönste, dass die Einführung dieses Kultes gerade dem König Tigranes zugeschrieben wird. Nach alledem, glaube ich, hat die Gleichung Ba'alšamīm = Baršamin kein ernsthaftes Bedenken gegen sich. Baršamin ist demnach der aramäische Himmels- und Weltgott. Sind aber die bisherigen Ausführungen richtig, so wirft das zugleich ein Licht auf die rein äußerliche Art, wie sich das armenische Pantheon gebildet hat. Auf königlichen Befehl wird eine Götterstatue aus ihrer syrischen Heimat entführt und bildet nun in Armenien sogleich den Mittelpunkt eines Kultes. Der Dienst wird also einfach von oben oktroyiert. Im Orient ist es zu allen Zeiten so gewesen. Die aus Arku vom elamitischen König entführte Statue wird in Susa das Centrum des Nanadienstes. An das äußere leibhaftige Bild schließt sich der Dienst an. Wie die Göttin Anahit in den Hymnen mit Rücksicht auf ihre goldene Statue verherrlicht wird, so empfängt auch von seinem Bildnis der Himmelsgott den Namen des Weißglänzenden. Besonders merkwürdig ist aber, dass der neu importierte Gott mit dem längst einheimischen Vahagn in nahe Beziehung tritt und dass diese Beziehung ihren mythologischen Ausdruck findet. Die mythenschaffende Kraft des Volkes muss demgemäß in König Tigranes' Zeit noch besonders lebendig gewesen sein. Das stimmt auch mit andren Beobachtungen. Die Zeit des Tigranes,\footnote{Nur die unzeitig angebrachte Gelehrsamkeit des Moses erkennt in dem Tigran der Heldenlieder den Tigranes der Cyropaedie, während das Volk natürlich den König der Könige feierte.} seiner unmittelbaren Vorgänger und seiner Nachfolger wird in den epischen Gesängen besonders gefeiert. Diese Lieder enthalten zeitgenössisch richtige, von den späteren nicht mehr richtig verstandene Erinnerungen, so z. B. an den Feldzug des Domitius Corbulo. Demnach kann ihre Entstehung zeitlich nicht allzufern von den besungenen Ereignissen liegen. Da hat die Annahme nichts unwahrscheinliches, dass die dichterische Schaffenskraft der Tigranesperiode auch dem mythologischen Liede zu Gute gekommen sei.

Ebenfalls syrischen Ursprung ist wohl Astłik. Ihr Haupttempel ist in Aštišat, wie die schon erwähnte Stelle des Agathangelos S. 499 beweist: „Der dritte Tempel hieß der der Göttin Astłik (eigentlich des astłikischen Abgotts \begin{armenian}Աստղկան դից\end{armenian}), das Schlafgemach des Vahagn nach dem Griechischen genannt, welche Ap̔roditēs\footnote{Agathang. S. 469 \begin{armenian}ռր է ինքն Ափրոդիտէո\end{armenian}. Lagarde Agathangelus S. 138 bemerkt: Also hat das armenische „Original“ die griechische Kasusendung erhalten, was nicht für Originalität spricht. Die armenische Übersetzung des Mar Michael S. 134 hat ebenfalls bezüglich der auf Christi Grab errichteten Aphroditestatue: \begin{armenian}պատկեր Աստղկայ որ է Ափրոդիտէս\end{armenian}. Indessen hat der Übersetzer vielleicht so nur mit Berücksichtigung der Agathangelosstelle geschrieben.} selbst ist.“ Über die Einsetzung ihres Dienstes unter König Tigranes berichtet Moses von Choren 2. 14: Aber die Bildsäule der Aphrodite (\begin{armenian}զԱպրոդիտայ զպատկերն\end{armenian}), als der Geliebten des Herakles befahl er neben der Bildsäule des Herakles selbst aufzustellen an den Stätten der Opfer. (\begin{armenian}յաշտից տեղիսն\end{armenian}). Thomas Arcruni (S. 54 Ausg. von St. Petersburg 1887) bemerkt, dass die Gemahlin des Königs Artašēs Sat͑inik „nicht abließ von dem Götzendienst der Astłik genannten Bildsäule.“\footnote{\begin{armenian}ոչ ՚ի բաց մերժեցտւ ՚ի կրոցն Աստղկայ անուանեալ պատկերին\end{armenian}.} Die sonstigen ziemlich häufigen Erwähnungen der Astłik bei den Chronographen sind nur Übersetzungen des griechischen \begin{greek}Ἀφροδίτη\end{greek} oder \begin{greek}Ἀστερία\end{greek}. Über den Namen hat G. Hoffmann die wohl richtige Vermutung ausgesprochen,\footnote{Akten S. 136.} dass Astłik „Sternchen“ eine Übersetzung des syrischen Kaukabtā sei, „dass nicht sowohl „Sternin“ als vielmehr „kleiner Stern“ bedeutet, den Planeten Venus „das kleine Glück“ zum Unterschiede von Jupiter, „dem großen Glück.“ Demnach wäre Astłik die Armenisierung der syrischen Göttin Bēltī. Unterstützt wird diese Vermutung durch die Lage ihres Heiligtums in der ehemals syrischen, dann armenisierten Provinz Taraunitis. Vahagn als Repräsentant der armenischen Eroberer tritt in nahe Verbindung zur autochthonen Landesgöttin.

Von derselben Seite ist wohl auch die Göttin Nana-Nanai ins armenische Pantheon übernommen worden. Über diese, später von Elamiten wie den Semiten gleich eifrig verehrte Gottheit der chaldäischen Urbevölkerung hat in einer erschöpfenden Monographie G. Hoffmann\footnote{G. Hoffmann Akten S. 130-161.} gehandelt. Was nun den speziell armenischen Kult dieser Göttin betrifft, so liegt ihr Sitz in dem westlichen, ehemals semitischen, von Kataonern bevölkerten Landstriche. Ihr Name wird nur in dem lakonischen Berichte des Agathangelos S. 457 erwähnt: „Von dort setzten sie auf das jenseitige Ufer des Flusses Gayl (Lykos) hinüber, zerstörten den Tempel der Nanaea, der Tochter des Aramazd, in dem Flecken T͑iln. Und nachdem sie die Schätze beider Tempel\footnote{Nämlich der Anahit von Erēz und der Nanaea von T͑iln.} geraubt und zusammengebracht hatten, überließen sie dieselben als Opfergabe dem heiligen Dienste der Kirchen Gottes mitsamt den Ländereien.“ Der Grieche hat \begin{greek}Ἀθηνᾶς βωμόν\end{greek}, und dieselbe griechische Bezeichnung gebraucht auch Moses von Chorēn für die Göttin. Sie scheint demnach in Armenien im Gegensatz zu Anahit der mütterlichen Gottheit mehr den jungfräulich herben Charakter der Kriegsgöttin gehabt zu haben. Eigentümlich ist, dass die armenischen Göttinnen stets in ein Verhältnis zu den männlichen Gottheiten treten, Anahit und Nanaea zu Aramazd als Töchter, Astłik als Geliebte zu Vahagn.

Im Texte des armenischen Agathangelos haben wir, wie gewöhnlich, nicht den Namen der Göttin selbst, sondern lediglich das von dem Namen gebildete Adjektiv: \begin{armenian}Նանէական մեհեանն\end{armenian}, der Nanēanische Tempel. Daraus wird nun von den meisten Armenisten eine Göttin Nanē erschlossen. Allein nach der Analogie z. B. von \begin{armenian}Ասիա Ասիական\end{armenian} kann \begin{armenian}Նանէական\end{armenian} auch von \begin{armenian}Նանէա\end{armenian} (\begin{armenian}Նանէայ\end{armenian}) abgeleitet werden. Wenn das richtig ist, so würde mit \begin{armenian}Նանէա\end{armenian} nicht die orientalische, sondern die griechische Namensform der Göttin wiedergegeben. Die Armenier hätten dann die Göttin nicht von genuinen Orientalen, sondern durch hellenistische Vermittlung resp. durch bereits hellenisierte Syrer empfangen. Der Bericht über die Einführung des Dienstes bei Moses steht dem wenigstens nicht entgegen.

Den aramäischen Einfluss auf die religiöse Entwickelung der Armenier beweisen außer den aus Syrien entlehnten Götterdiensten noch einige weitere Umstände. Die Bezeichnung unsrer christlichen Quellen für die Priester ist \begin{armenian}Քուրմ\end{armenian} K͑urm (regelmässige Benennung für die Götzenpriester) und \begin{armenian}քահանայ\end{armenian} K͑ahanā. Beide Bezeichnungen hat Lagarde (armen. Studien S. 160 und 157) aus dem syrischen hergeleitet. Das deutet auf starke Beeinflussung des armenischen Götterdienstes durch die Syrer.

Des Ferneren hat sich in der sagenhaften Überlieferung der Armenier eine deutliche Erinnerung an den starken, ursprünglich feindlichen Einfluss der syrischen Kultur und Religion erhalten. Wie die Kaunier jährlich gewappnet mit Schild und Speer in feierlichem Umzuge die fremden Götter von den Landesgrenzen zurückweisen, so bewahrt auch die altarmenische Tradition das Andenken an den Kampf des durch den \begin{greek}ἥρως ἐπώνυμος\end{greek} Hayk repräsentierten ursprünglichen Volksbewusstseins mit dem syrischen, als dessen Vertreter der zum Urkönig euhemerisierte Titan Bel erscheint. Der Kampf des Hayk mit Bel ist in zwei Rezensionen erhalten, einmal bei Moses von Chorēn, sodann in dem sogenannten Pseudoagathangelos, welcher sagt er schöpfe: \begin{armenian}՚ի մատեանն Մարաբայ փիլիսոփացի Մծուրնացւոյ\end{armenian} „aus dem Buche des Mar Abas des Philosophen von Mcurk͑.“\footnote{Sebēos ed. Patkanean S. 1; für \begin{armenian}Մծուրնացւոյ\end{armenian} hat Baumgartner die scharfsinnige, von Vetter und Carrière gebilligte Konjektur \begin{armenian}Մծուինացւոյ\end{armenian} des Nisibeners gemacht. ZDMG 1886 S. 495. Nötig ist sie nicht. Mar Abas stammte aus Mžurk͑ Mcurk͑ = \begin{greek}κλίμα Μουζουρῶν\end{greek}. Georg. Cypr. S. 183. In Mar Abas Zeit schrieben auch die Armenier nur syrisch oder griechisch, und wenn Moses denselben mit Recht einen Syrer (\begin{armenian}Սսռրի\end{armenian} 1. 8) nennt, so waren diese damals gerade in den südarmenischen Gauen zahlreich vertreten.} Es kann nach den Ausführungen von Gutschmid\footnote{Kl. Schr. 3. S. 320 ff.} und namentlich von P. Vetter\footnote{Festgruss an R. Roth 1893 S. 81 ff.} als feststehend betrachtet werden, dass wir hier tatsächlich einen Auszug aus Mar Abas Katinā, der Quelle des Moses, vor uns haben; nur gibt der Auszug den wirklichen Inhalt wieder, während Moses sich starke Bearbeitungen und allerlei willkürliche Änderungen erlaubt. Ich gebe deshalb das Land vom Kampfe zwischen Hayk und Bel nach dem Auszuge des Unbekannten\footnote{Sebēos a. a. O. S. 3.}:

„Zu jener Zeit herrschte in Babelon der riesenhafte Jäger, Bēl der Titanide (\begin{armenian}Տիտանեան\end{armenian}) der Prächtige, welcher unter die Nicht-Götter gerechnet ward. Er war von gewaltiger Kraft und sein Hals von großer Schönheit. Er war der Fürst aller Völker, welche über das Antlitz der ganzen Erde zerstreut waren. Dieser machte durch zauberisches Blendwerk Trugkünste und (erließ) königliche Befehle an alle Völker. In seinem Stolze und seiner Hoffart errichtete er sein Standbild, und er gebot ihm Anbetung, wie Gott, zu erweisen und Opfer darzubringen.

Und alsobald vollzogen alle Völker seine Befehle. Nur ein einziger Häuptling der Völker, Hayk mit Namen, unterwarf sich nicht seiner Dienstbarkeit; nicht errichtete er in seinem Hause das Standbild desselben, noch verehrte er ihn mit göttlicher Ehre. Hayk war sein Name; gegen ihn erhob sich des Königs Bēl gewaltiger Grimm. Ein Heer versammelte König Bēl in Babelōn; er stürmt, er zieht aus gegen Hayk, um ihn zu töten.

Er zieht aus, er gelangt nach dem Lande Ararad in das Haus, welches ihr väterliches (Gut) war, welches am Bergesfuß gebaut war. Und Kadmos kam flüchtigen Fußes nach Hark͑ zu seinem Vater, ihm die Nachricht zu bringen und sagt: Stürmend zieht Bēl der König gegen dich aus; er ist gekommen, angelangt bis zu dem Hause dort. Und ich mit meiner Frau und meinen Söhnen bin hier gleich einem Flüchtling.

Es nahm Hayk den Aramenak und den Kadmos, seinen Sohn, und die Söhne derselben und die Söhne seiner sieben Töchter, riesige Männer, doch gering an Zahl. Er tritt entgegen Bēl dem Könige; nicht konnte er Stand halten vor der Menge der riesigen gepanzerten Männer.

Da tritt Hayk dem Bēl entgegen und Bēl wollte ihn greifen mit eigner Hand. Rückwärts wandte sich Hayk vor seinem Angesicht und als Flüchtling zog er davon. Bēl setzte hinter ihm her eifrigste zusammen mit seinem Waffenknecht.

Halt machte Hayk und er spricht zu ihm: „Warum so eifrig stürmst Du hinter mir her? kehre von hier zurück nach Deinem Orte, damit Du nicht diesen Tag umkommest durch meine Hände. Denn mein Pfeil verfehlt niemals sein Ziel.“

Antwort gab ihm Bēl und spricht: „Eben darum, damit Du nicht in die Hände meiner Jungmannschaft fallest und umkommest. Vielmehr komm in meine Hände und lebe in Frieden in meinem Hause, um zu ihren Verrichtungen die Jägerburschen in meinem Hause anzuhalten.“\footnote{Der waffenkundige Gebirgssohn soll also Oberjägermeister beim König von Babylon werden.}

Antwort gab ihm Hayk und spricht: „Ein Hund bist Du und von einer Hundemeute (stammt ihr) Du und Dein Volk. Deshalb fürwahr! werde ich heute meinen Köcher gegen Dich ausschütten.“ Und der König, der Titanide, gepanzert, verließ sich auf seine starke Bepanzerung. Und Hayk, Abets Spross, nahte sich ihm und hielt in seiner Hand den Bogen gleich einem riesigen Balken von Cedernholz. Und Hayk, seinen Platz behauptend machte sich jenen gegenüber mit dem Bogen bereit.

Und er richtete auf den Boden die Köcher mit ihm selbst dem Ringe\footnote{\begin{armenian}յաւղ\end{armenian} die Hndn; Patkanean liest dafür \begin{armenian}յաղեղն\end{armenian}. Meine Hilfsmittel reichen zur Erklärung dieser dunkeln Stelle nicht aus. Auch in Moses' Parallelbericht (1. 11) ist Hayk's Meisterschuss der Mittelpunkt des ganzen Kampfberichtes.} des riesenmäßigen Bogens, und spannend mit Kraft den Bogen, verwundete er mit dem Pfeil die Planke von Eisen; durch den Schild von Erz hindurch durchbohrte er den Koloss von Fleisch; auf den Erdboden spießte ihn der abgesandte Pfeil. Alsbald auf die Erde warf er den Riesen, der sich für Gott hielt. Die Heere desselben wurden flüchtig. Und jene hinter ihnen herdringend, schleppten weg ihre Herden von Pferden, Mauleseln und Kamelen.

Und Hayk kehrte nach seinem Orte zurück. Es zog aus, nahm in Besitz Hayk das Land Ararad. Und daselbst wohnte er mit seinem Volke bis jetzt. In der Stunde seines Todes gab er sein Patrimonium zur Erbschaft dem Kadmos, seinem Enkel, dem Sohne des Aramaneak, dem Bruder des Harmā.“

Mit einem genuin heidnischen Bruchstück haben wir es hier nicht zu tun. Der Eingang zeigt deutlich die Spur christlicher Bearbeitung; das Gebot göttlicher Verehrung des babylonischen Königs ist sichtlich Nachahmung der Erzählung des Danielbuches. Allein der nachfolgende Bericht mit den echt epischen Schimpfreden der beiden Helden macht einen durchaus altertümlichen und ursprünglichen Eindruck. Auch die Herabdrückung des Gottes Bēl zu einem Helden der Vorzeit braucht nicht erst das Werk einer christlichen Bearbeitung zu sein. Es ist der Rationalismus, der uns aus Sueton und Philo von Byblos geläufig ist. Das Beiwort Bēls \begin{armenian}ճոխն չաստուածացեալ\end{armenian} „der Gewaltige unter die Nicht-Götter gerechnete“ ist christliche Korrektur eines \begin{armenian}աստուածացեալ\end{armenian}, „unter die Götter gerechnet.“ Dann rührt aber die ursprüngliche Fassung von einem Nichtchristen her.

Noch ein zweiter Sagenbericht ist ein Beleg für den syrischen Einfluss, die Sage von Semiramis (Šamiram); es ist die Erzählung von ihrer Liebe zu Arā (Aray) Arams Sohn. Ich gebe dasselbe gleichfalls in der Version des Auszugs bei Sebēos (a. a. O. S. 5):

„Und Šamiram, die Gemahlin des Ninos, des Königs der Assyrer, als sie hörte von seiner Schönheit, wünschte Freundschaft mit ihm zu schließen, um ihren Gelüsten durch Hurerei\footnote{\begin{armenian}պոռնկութիւն\end{armenian} fornicatio. Das griechische Wort ist wohl christlicher Ersatz für einen ursprünglich weniger harten Ausdruck.} Genüge zu leisten. Denn auf die bloße Nachricht hin ward sie sehr verliebt in ihn selbst und in seine Schönheit. Denn überhaupt nicht ward gefunden zu jener Zeit ein Mann ihm gleich, ein (so) bildschöner Mann. Sie schickt zu ihm Boten mit Geschenken und ruft ihn zu sich nach Ninuē. Aber Arā nimmt ihr Geschenk nicht an und gewinnt es nicht über sich, nach Ninuē zur Šamiram zu gehen. Da nimmt ihre Heerestruppen Šamiram und zieht gegen ihn aus nach Armenien. Sie marschirt, gelangt in die Ebene von Arā und liefert eine Schlacht dem Arā; sie schlägt das Heer und tötet den Arā in der Schlacht. Da befiehlt Šamiram seinen Leichnam auf den Söller ihres Palastes zu schaffen und spricht: „Ich werde den Göttern gebieten seine Wunden zu lecken und er wird wieder aufleben.“

Aber als sein Leichnam auf dem Söller zu faulen begann, befahl sie ihn heimlich in eine Erdspalte niederzulegen und zu verbergen. Sie schmückte einen von ihren Lieblingen, einen wohlgebildeten Mann. Und sie verbreitete die Nachricht, dass die Götter den Arā geleckt und auferweckt hätten. Sie hielt ihn im Versteck und keinem ihrer Bekannten zeigte sie ihn. Und so sprengte die Herrin Šamiram das Gerücht (oder die Legende) von den Aralez aus. Damals gebot Šamiram über das Land der Hayastanier, und von da herrschten die Könige von Assyrien (Asorestan) bis zum Tode Senek͑arim's. Dann machten sie sich frei von dem Joche der assyrischen Könige.

Was der weitläufige Bericht des Moses mehr enthält, ist lediglich griechischen Quellen entnommen. Höchstens das Sprüchwort kann hier angeführt werden\footnote{Mos. Chor. 1. 18.}: \begin{armenian}Ուլունք Շամիրամայ ՚ի ծով։\end{armenian} „Die Perlen der Semiramis ins Meer.“ Dem aufmerksamen Beobachter wird nicht entgehen, dass auch diese Sage keineswegs genuinen syrisch-armenischen Götterglauben enthält. Der Name der Semiramis und ihres Gatten gehen auf Ktesias zurück, den z. B. Julius Africanus, der Vater des christlichen Euhemerismus, notorisch benutzt hat. Aber muss deshalb diese Legende erst von Christen fabriziert oder wenigstens zugestutzt worden sein? Nötig ist die Annahme entschieden nicht. Auch der Euhemerismus des Africanus ist großenteils aus Philochoros, Palaephatos oder Suetonius entlehntes Gut. Wenn auch Mar Abas im Geiste der christlichen Apologeten manches zugesetzt oder zurechtgemacht haben mag, so kann er doch auch bereits eine heidnische Quelle benutzt haben, welche die einheimischen Mythologeme mit ktesianischer und ähnlicher Weisheit versetzte; war doch euhemeristische Mythenerklärung damals Gemeingut aller höher Gebildeten und nicht zum wenigsten der \begin{greek}ἱερεῖς\end{greek} und \begin{greek}μάντεις\end{greek}. Ich glaube deshalb nicht, dass alle die rationalistischen Wendungen dieser und verwandter Erzählungen den Christen dürfen ohne weiteres zugeschoben werden. Sie können recht gut bereits in einer heidnischen Quelle vorhanden gewesen sein.

Dass wir es in der Semiramislegende mit einem ursprünglich aramäischen Mythus zu tun haben, zeigt das Auftreten der Semiramis, welches vollkommen dem der Liebesgöttin Ištar im babylonischen Epos verwandt ist. Ihr Liebling ist der Sohn des Aram, des \begin{greek}ἥρως ἐπώνυμος\end{greek} der Aramäer, welchen die Armenier mit großer Naivetät sich angeeignet und zum Nationalheros umgeschaffen haben. Eusebius in der Chronik z. 1. Abr. 400 gedenkt des erdgebornen Syros: \begin{greek}τούτοις τοῖς χρόνοις Σύρος ἱστορεῖται γεγονέναι γηγενής, οὗ ἐπώνυμος ἡ Συρία\end{greek}. Sync. 283, 12. Die weibische Schönheit Arās erinnert an Adonis, Dūzi, Attis und die übrigen Lieblinge der großen Mutter. Aber Arā ist kein liebend sich hingebender Adonis; vielmehr weist er Semiramis' Liebesanträge schroff zurück, und das führt seinen Untergang herbei. Hier findet sich die genaue Parallele in jenem Abschnitte des Gilgamēš-Epos,\footnote{Vgl. G. Smiths chaldaeische Genesis (deutsche Ausgabe von Fr. Delitzsch) 1876 S. 189 ff.} wo der Held die Liebesanträge der Göttin verwirft und das Schicksal aller derjenigen aufzählt, welche durch Ištars Liebe zu Grunde gegangen sind. Indessen auch hier hat eine Kontamination stattgefunden. Der Liebling der syrischen Göttin, der Sohn des Syrers, hat seinen ursprünglichen Namen nicht behalten, sondern ist durch den Nationalarmenier \begin{armenian}Արայ\end{armenian} Arā ersetzt worden. Bereits Emin und ihm folgend Langlois\footnote{Collection des hist. a. et m. de l'Armén. 1. S. Note. Vgl. jetzt auch J. Marquart ZDMG 49 S. 658 ff.} haben ihn mit dem platonische Er zusammengestellt... \begin{greek}Ἠρὸς τοῦ Ἀρμενίου τὸ γένος Παμφύλου · ὅς ποτε ἐν πολέμῳ τελευτήσας, ἀναιρεθέντων δεκαταίων τῶν νεκρῶν ἤδη διεφθαρμένων, ὑγιὴς μὲν ἀνῃρέθη, κομισθεὶς δὲ οἴκαδε, μέλλων θάπτεσθαι, δωδεκαταῖος ἐπὶ τῇ πυρᾷ κείμενος ἀνεβίω\end{greek}. Plato Republ. 10. p. 614 B. Ohne Zweifel hatte der alte Mythus erzählt, dass Arā tatsächlich wieder aufgelebt sei, und es ist lediglich späterer Euhemerismus, wenn darin ein Gaukelspiel der Königin erblickt wird, das obendrein missrät. Bei Moses Ch. 1. 15 sagt Šamiram: \begin{armenian}հրամայեցի աստուածոցն իմոց լեզուլ զվէրս նորա, և կենդանասցի\end{armenian}. „Ich habe meinen Göttern befohlen, seine Wunden zu lecken, und er wird wieder aufleben.“ Ganz ebenso bei dem Anonymus (Sebēos S. 5): \begin{armenian}Ես ասացից աստուածոց լիզուլ զվէրս նորա և կենդանասցի\end{armenian}. „Ich werde den Göttern sagen seine Wunden zu lecken und er wird wieder aufleben. Zum Schlusse fügt er noch bei: \begin{armenian}և այնպէս հանէ համբաւ Արալեզաց տիկինն Շամիրամ\end{armenian}. „Dergestalt verbreitete die Königin Šamiram die Sage von den Aralez.“ Diese Aralez sind eine uralte einheimische Vorstellung, welche dem armenischen Geisterkult angehört. Eznik von Kołb (um 450) in seiner Widerlegung der verschiedenen Heidensekten gedenkt des merkwürdigen Mythus der Abstammung dieser Geister von einem göttlichen Hunde (Ausgabe von Smyrna 1760 S. 88): „Nicht von einem Hunde stammt irgendein Wesen ab; (sie erzählen) als wenn dasselbe (ausgestattet) mit unsichtbaren Kräften lebe und wenn ein Verwundeter im Kampfe fällt und daliege, ihn lecke und heilt. Doch das Alles sind Fabeln und Altweibergewäsch und in der Hauptsache (hervorgegangen) aus der Irrlehre der Dämonen.“ Offenbar ist auch der platonische Er durch die Tätigkeit dieser Geister wieder zum Leben erweckt worden. Wie zäh dieser Glaube bei den Armeniern haftete, zeigt eine Erzählung des Faustus aus tief christlicher Zeit. Als im Jahre 375 Mušeł der Mamikonier, der Krongrossfeldherr (Sparapet) von Armenien, den Streichen der von König Varaztad gedungenen Bravi erlegen war, legten die Gentilen den abgehauenen Kopf mitsamt der Leiche auf die Zinne des Schlossthurms mit den Worten\footnote{Faustus 5. 36.}: „Weil er ein tapferer Mann war, werden die Aṙlez hinabsteigen und ihn auferwecken.“ Erst als der Leichnam zu verfaulen beginnt, ist ihr felsenfester Glaube erschüttert. Dass die Familie, welche an der Spitze der antiköniglichen feudal-klerikalen Adelsparthei stand, noch so tief im Heidentum steckte, musste in dem bereits völlig christianisierten Lande das größte Aufsehen machen. Gewiss haben die Reden der Mönche und heiligen Lehrer dieses schlagende, allen vor Augen gestellte Beispiel von der Ohnmacht der Heidengötter mit Vorliebe zum Gegenstand ihrer erbaulichen Reden gewählt. Man beachte nun die auffällige Parallele zwischen dem Mar Abas-Berichte und dem Faustus-Berichte. Beidemale wollen die nächsten Angehörigen an den Tod des Erschlagenen nicht glauben; beidemale findet eine ähnliche Ausstellung des Todten an erhöhter Stelle statt. Beidemal führt erst die eingetretene Verwesung zur Bestattung der Leiche. Nun ist es gewiss nicht zufällig, dass die Abfassung der Schrift des Mar Abas Katinā wenig später fällt, als das so auffällige Ereignis im Mamikonierhause. Als Abfassungszeit jenes Geschichtswerkes ist nämlich von Gutschmid\footnote{Kl. Schr. 3. S. 321 vgl. 334.} die Zeit um 383-388 festgesetzt worden.\footnote{P. Vetter a. a. O. schreibt die dem Mar Abas-Berichte angehängte Liste der armenischen Könige einer anderen Quelle zu, weil sie nach Moses von Chorēn neben Aršak dem Jüngern den Vałaršak als Mitregenten aufzählt. Die Worte \begin{armenian}էղբարբ\end{armenian} — \begin{armenian}աշխարհիս\end{armenian} Sebēos S. 10 sind als Glossem des Interpolators auszuscheiden. Dann fällt das Hauptbedenken gegen die Zuweisung der Liste an Mar Abas.} Ohne Zweifel hat der christliche Bearbeiter nach seinen unmittelbaren Erfahrungen von der Nichtigkeit der göttlichen Aralez den Schluss der Sage demgemäß bearbeitet und die unschöne Wendung von dem verfaulenden Leichname hineingesetzt.

Wenn wir diese euhemeristischen Zuthaten entfernen, bleibt als Residuum der Mythus von der syrischen Liebesgöttin mit ihrem Liebling zurück. In der Sage von Šamiram hat sich die Legende von der Astłik erhalten. Man darf wohl annehmen, dass an ihrer ursprünglich syrischen Kultusstätte Aštišat sich im Liede dem unglücklichen Ausgang ihres ersten Liebeshandels, wohl der Gesang von dem \begin{greek}ἱερὸς γάμος\end{greek} mit dem Vahagn-Herakles angeschlossen hat.
\clearpage
\section{Die national-armenischen Gottheiten}
\paragraph{}
Baršamin, Astłik, Nanēa, die Legenden von Bēl und von Šamiram repräsentieren den Einfluss, welchen das semitische Element auf Armenien ausgeübt hat. Es bleiben die Götter national-armenischen Ursprungs.

Hier ist vor allem Gott Vanatur zu erwähnen. Wiederum berichtet über seinen Kult Agathangelos, eine Angabe, welche darum von Interesse ist, weil sie uns zeigt, in welcher Form die heidnischen Feste durch christliche ersetzt wurden. Eine so populäre und wichtige Feier, wie das Neujahrs- und Erntefest erhielt durch die aus Kaisareia mitgebrachten Heiligenreliquien einen ausgesprochen christlichen Charakter. Da die Worte mehrfach Schwierigkeiten enthalten, setze ich sie her Agath. S. 482: \begin{armenian}Եւ զյիշատակս վկայիցն բերելոց ժամադրեաց տօն մեծ հռչակել, յառաջագռյն կարծեալ սնոտեացն պաշտաման ՚ի ժամանակի դիցն Ամանորայ ամենաբեր նորոց պտղոց տօնից հիւրընկալ դիցն Վանատրի զոր յարաջագոյն իսկ ՚ի նմին տեղւոջ պաշտէին՝ յուրախութեան Նաւազարդ աւուր։\end{armenian} Einen mehrfach abweichenden Text muss der Grieche vor sich gehabt haben § 149: \begin{greek}καὶ τὰ μνημόσυνα τῶν ἐνεχθέντων μαρτύρων ἔταξεν εἰς τὴν μεγάλην πανήγυριν τῆς διαπομπῆς, τῆς ματαίως εἰς τιμὴν τῶν παλαιῶν σεβασμάτων γενομένης ἀπὸ τῶν καιρῶν τῶν νέων εἰς τὰς ἀπαρχὰς τῶν καρπῶν, ξενοδεκτῶν θεῶν λεγομένης τῆς πανηγύρεως ἣν ἐπιτελοῦσιν ἐν τῷ τόπῳ ἐκείνῳ εὐφραντικῶς ἀπὸ τῶν ἀρχαίων καιρῶν ἐν ἡμέρᾳ τῆς πληρώσεως τοῦ ἐνιαυτοῦ\end{greek}. In dem armenischen Texte ist zu \begin{armenian}՚ի ժամանակի դիցն Ամանորայ\end{armenian} „zur Zeit des Gottes Amanor,“ Apposition \begin{armenian}տօնից հիւրընկալ դիցն Վանատրի\end{armenian} „der Feste des gastfreundlichen Gottes Vanatur.“ Wie heißt demnach der Gott? Amanor oder Vanatur? Eine so durchsichtige Personifikation, wie „Gott Neujahr,“ erinnert eigentlich mehr an römische Abstraktionen, als an das im armenischen Kulte sonst übliche. Man vergleiche parallele Stellen z. B. S. 455: da wollen sie zuerst zerstören \begin{armenian}զսուտ աստուածոցն զբագինսն\end{armenian} „die Heiligtümer der falschen Götter“ und das wird dann genauer definiert als \begin{armenian}մեհեան\end{armenian}... \begin{armenian}սպիտակափառ դիցն Բարշիմնիա։\end{armenian} „Tempel des weißglänzenden Gottes Baršimnia. Ganz ebenso erwähnt an unsrer Stelle der Grieche zuerst die \begin{greek}παλαιὰ σεβάσματα\end{greek} im Allgemeinen und dann speziell die \begin{greek}φιλόξενοι θεοί\end{greek}. Es ist wohl mit einer leichten Umstellung des Wortes \begin{armenian}դիցն\end{armenian} zu schreiben: \begin{armenian}յառաջագոյն կարծեալ սնոտեացն դիցն ՚ի ժամանակի ամանորայ\end{armenian} u. s. f. Die Worte \begin{armenian}ամենաբեր\end{armenian} — \begin{armenian}պտղռց\end{armenian} sind dann mit \begin{armenian}Վանատրի\end{armenian} zu verbinden. „Die Gedächtnisstage der (von ihm) transferierten Märtyrer bestimmte er als großes Fest weihevoll zu begehen, während früher der Dienst der nichtigen Götter gegolten hatte zur Zeit des Neujahrs, der Festfeier des Allbringers der neuen Früchte, des gastfreundlichen Gottes Vanatur, welchen sie früher an eben diesem Orte freudvoll verehrt hatten am Navasardtage.“

Einen wichtigen Beitrag zur Lehre von Gott Vanatur liefert Moses von Chorēn angeblich aus Bardesanes' (Bardacans) Fortsetzung der Tempelgeschichte des Olympios von Ani 2. 66: ...Er erzählt in (der Geschichte) von den Tempelriten, dass der letzte Tigran, der König von Armenien, um das Grabmal seines Bruders, des Oberpriesters Mažan zu ehren, in der Burg der Altäre im Gau Bagrevand einen Altar\footnote{oder Sanktuarium.} über dem Grabe gebaut habe, damit an den Opfern alle Pilgrime fröhlich sich beteiligten und die Fremdlinge ein Nachtlager erhielten. In der Folgezeit hat Vałarš ein allgemeines Landesfest eingerichtet im Beginn des neuen Jahres, beim Eintritt des Navasard.“\footnote{Auf dieses Neujahrsfest \begin{armenian}առաւօտն նաւաոարդի\end{armenian} den frohen Morgen des Neujahrs bezieht sich das einzige Lied aus Gołt͑n, welches wir nicht dem Mosēs, sondern dem Gregorios Magistros verdanken. Sein Verfasser soll Ardašēs der Parther sein. Journ. Asiat. 6. Sér. T. 13. S. 53.} Man sieht, die Armenier sind gelehrige Schüler der griechischen Euhemeristen. Wie diese ein Zeusgrab in Kreta, ein Dionysosgrab in Delphoi und zahlreiche andere Gräber der Götter genannten Menschen nachweisen, so wird auch hier Vanaturs Heiligtum in Bagavan zu dem Grabe eines Oberpriesters aus dem Königsgeschlecht gemacht. Dabei kann allerdings historisch sein, dass Mažan der K͑urmapet tatsächlich an dieser hochheiligen Stätte sein Begräbnis erhielt, und das Fest des Gottes Vanatur erst durch König Vałaršak aus einem lokalen Kantonalfest zu einer allgemeinen offiziellen Landesfeier erhoben wurde.

Ein Beweis für die Heiligkeit von Bagavan ist, dass Gregor gerade hier, wie in Aštišat, dem zweiten altheiligen Centralpunkt der Heiden, die aus Kaisareia mitgebrachten Reliquien deponiert. In der syrischen Kirche, mit der, wie die Übertragung der Abgarlegende nach Armenien erweist, die armenische Kirche früh und nahe verbunden war, ist schon sehr bald der Reliquienkultus zu großer Blüthe gelangt; man denke an die Translation der (vielleicht echten) Leiche des Apostels Thomas nach Edessa bereits unter Kaiser Alexander Severus. Den Reliquien wird Schutzkraft gegen die teuflischen Dämonen zugeschrieben. Der Apollo von Daphne kann unter Julian nicht weissagen, weil dem Tempel gegenüber die Märtyrerkapelle des Babylas errichtet ist.\footnote{Socrates h. e. 3. 18, Sozomenus 5. 19.} Besonders interessant ist hierfür eine Äußerung des Isaak von Antiochien\footnote{S. 154 in den Übers. von Bickell.} in seiner Belehrung über die Teufel: „Wenn jemand schwört oder schwören lässt, so scheint uns die Kirche zu gering, um darin den Eid anzunehmen. Der Schwörende wünscht den Eid in der Kirche zu leisten; aber der Empfänger will ihn dort nicht annehmen, indem er sagt: „Wenn Du nicht in der Kapelle des Apostels Thomas schwörst, so traue ich Dir nicht.“ Weil dort die Dämonen heulen, ehren sie den Apostel mehr, als seinen Herrn. In der Kirche ist man gleich bereit zu schwören, aber in den Apostelkirchen zögert man lange damit.“ Also nach syrischer Volksanschauung war ein Schwur bei den Märtyrern kräftiger oder schauerlicher, als bei Christus. Isaak gibt dazu eine theologische Erklärung. Die Apostel und Märtyrer sind nach ihm mit der Züchtigung der Dämonen betraut, während die Würde Christi den letzteren verbietet, Lärm in seiner Kirche zu machen. Auch der h. Gregor teilt durchaus diese volkstümliche Anschauung; die Märtyrergebeine, welche er mit so großer Feierlichkeit in Aštišat und Bagavan beisetzte, hatten das Recht zur Züchtigung und die Kraft zur Vertreibung der altansässigen heidnischen Dämonen.

Zu dem echt armenischen nationalen Kult wird wohl auch die Verehrung von Sonne und Mond zu rechnen sein, welche Moses von Chorēn 2. 77 erwähnt. Er berichtet von Artašir, dem sāsānidischen Eroberer Armeniens: „Er vermehrte noch die Tempelzeremonien. Dann befiehlt er auch, dass das Ormizdfeuer auf dem Altare zu Bagavan unauslöschlich brenne. Aber die Bildsäulen, welche Vałaršak in Armavir errichtet hatte, die Abbilder seiner Ahnen mitsamt der Sonne und dem Monde, und welche nach Bagaran und hinwiederum nach Artašat versetzt worden waren, diese zerschlug Artašir.“ Artašir ist in der Chronologie des Moses an die Stelle seines Sohnes Šapuh (Šapūr) getreten. Dass er dabei das hochgefeierte Zentralheiligtum von Bagavan in ein \begin{greek}πυραιθεῖον\end{greek} verwandelte, ist charakteristisch; die Mazdayasnier haben also, wie die Christen, vor allem die Okkupation dieser Stätte für wichtig zur religiösen Eroberung des Landes angesehen. Dass die Perser die Götterbildnisse der Armenier zerschlugen, ist bei dem intoleranten Character der Zoroastrier wohl glaublich. Der armenische Polytheismus musste trotz seiner starken — freilich vielfach umgewandelten und nationalisierten — iranischen Elemente dem mazdayasnischen Könige als der reinste Götzendienst erscheinen. Eigentümlich ist nur an dem Berichte, dass der Perserkönig gerade die Bilder der Sonne und des Mondes soll zerschlagen haben. Hat er Anahit, Vahagn, Baršamin u. s. f. geschont? An und für sich ließe es sich sehr wohl denken, dass diese beiden, neben den königlichen Bildsäulen aufgestellt, die ursprüngliche armenische Religion, Ahnendienst und Verehrung der Lichtgottheiten repräsentierten. Aber auffällig ist doch, dass Agathangelos in seinem so genauen Berichte von der Zerstörung der Heidentempel dieser Sonnen- und Mondbilder gar nicht gedenkt. Man könnte ja zur Noth annehmen, dass in der kurzen Spanne Zeit, wo im wiederhergestellten Reiche unter Tiridates das Heidentum noch blühte, keine Möglichkeit zur Wiederaufrichtung dieser Bildsäulen oder sacella gegeben war. Indessen das bleibt immerhin eine bedenkliche Annahme. Und so wage ich weder den Bericht ausdrücklich zu verwerfen, noch unbedingt anzunehmen.

Zu den genuin armenischen Bestandteilen des Götterglaubens gehören endlich auch die Spuren des Animismus, wie wir dieselben bereits bei Besprechung der Šamiramlegende in den Aralez nachgewiesen haben. Eben dahin gehören auch die \begin{armenian}հարք\end{armenian} „die Väter“ die vergötterten Ahnherren des Volkes. Hayk, aus seinem angeblichen ersten Wohnsitze im Lande Ararad auswandernd, gelangt in eine hochgelegene Ebene „und er nannte diese Ebene mit Namen Hark͑ (Väter) nach dem Namen der Hark͑.“ Offenbar ist dieser Ort eine hochheilige Stätte, der Sitz der mythischen eisten Volksgenossen, der vergotteten Genarchen.

Außerhalb unsrer Betrachtung lasse ich zwei Gottheiten, welche nur Zenob von Glak kennt. In seiner Geschichte von Tarōn gedenkt dieser Schriftsteller nämlich zweier sonst völlig unbekannter Götter des \begin{armenian}Գիսանէ\end{armenian} (Gisanē, des Langhaarigen) und des \begin{armenian}Դեմետր\end{armenian} (Demetr), welche mitsamt einer indischen Kolonie aus ihrer östlichen Heimat nach Armenien sollen verpflanzt und daselbst verehrt worden sein,\footnote{Zenob von Glak Ausg. v. Venedig 1839 S. 8 u. s. f.} bis der h. Gregor diesen Kult nach einer heißen Schlacht vernichtete. Agathangelos berichtet aber hierüber nicht nur nichts, sondern die ganze Geschichte steht mit dem von ihm erzählten im schroffsten Widerspruch. Gregor soll an der ehemaligen Kultstätte dieser Gottheiten die Reliquien Johannes des Täufers und des Athenogenes vergraben, daselbst das Kloster des Surb Karapet errichtet und Zenob als ersten Bischof-Abt eingesetzt haben. Allein dieses hochberühmte Klosterinstitut hat nachweislich nicht existiert, solange Aštišat im Besitze von Gregors Familie war und kann frühestens im 5. Jahrhundert von den Mamikoniern gegründet worden sein. Bei dem unhistorischen Charakter dieses späten Apokryphons sind diese Götter als einfache Erfindungen zu streichen. Gisanēs Bruder Demetr, der doch wohl der griechischen Demeter seinen Ursprung verdankt, erweckt von den Kenntnissen des Schriftstellers nicht eben eine vorteilhafte Vorstellung.
\clearpage
\section{Einfluss des Hellenismus}
\paragraph{}
Auch Armenien ist, wie die andern Länder des Ostens, früh von der hellenistischen Kultur beeinflusst worden. Die einheimische Überlieferung leitet die Einführung des Bilderdienstes aus griechischer Einwirkung her, und es ist beachtenswert, dass sie gerade die beiden bedeutsamsten Gestalten des großarmenischen Reichs Artašēs (Artaxias) den Gründer und Tigranes den Vollender desselben als hervorragend philhellenisch hinstellt. Freilich hat dieser Philhellenismus einen spezifisch orientalischen Character; griechische Götterbilder und Priesterschaften sollen nach Armenien transportiert und dort angesiedelt worden sein. Das stimmt gut mit dem überein, was wir von König Tigranes aus den abendländischen Quellen wissen. Er hat massenhaft die griechische oder graecisierte Stadtbevölkerung Kilikiens und Kappadokiens in seinem Reich, vorab in der neuen Hauptstadt Tigranokerta angesiedelt.\footnote{\begin{greek}Ἑλλήνων δὲ τὴν Μεσοποταμίαν ἐνέπλησε, πολλοὺς μὲν ἐκ Κιλικίας, πολλοὺς δ' ἐκ Καππαδοκίας ἀνασπάστους κατοικίζων\end{greek}. Plut. Lucull. 21 vgl. 14 und 29 \begin{greek}ἐκ δώδεκα ἐρημωθεισῶν ἱπ' αὐτοῦ πόλεων Ἑλληνίδων ἀνθρώπους συναγαγών\end{greek}. Strabo 11. 532 C. 539 C. Dio Cassius 36, 4.}

Bei Moses von Chorēn und in der griechischen Übersetzung des Agathangelos werden nun die armenischen Götternamen folgendermaaßen wiedergegeben:

\begin{table}[H]
    \centering
    \small
    \begin{tabular}{l l}
        Aramazd  &  \begin{greek}Ζεύς\end{greek}      \\
         Mihr     &  \begin{greek}Ἥφαιστος\end{greek}  \\
         Anahit   &  \begin{greek}Ἄρτεμις\end{greek}   \\
         Nanēa    &  \begin{greek}Ἀθηνᾶ\end{greek}     \\ 
         Astłik   &  \begin{greek}Ἀφροδίτη\end{greek}  \\
         Tiur     &  \begin{greek}Ἀπόλλων\end{greek}   \\
         Vahagn   &  \begin{greek}Ἡρακλῆς\end{greek} \\
    \end{tabular}
\end{table}
\paragraph{}
Für Baršamin, dessen fremde syrische Herkunft in der Erinnerung blieb und für Vanatur existiert keine Übersetzung. Diese völlig feststehende und zum Theil wenig passende Korrespondenz (man vergleiche Mihr — \begin{greek}Ἥφαιστος\end{greek}) lässt darauf schließen, dass dieselbe längst in Übung bestand und von den Schriftstellern dem herrschenden Volksgebrauch entlehnt ward. Über die Einführung des griechischen Bilderdienstes hat nun Moses zwei völlig verschiedene Berichte, die offenbar auf einen verschiedenen Ursprung zurückgehen. Auch in der späteren Geschichte lassen sich sowohl mit dem ersten, wie mit dem zweiten Berichte zusammenhängende Anhänge nachweisen.
\begin{center}
\emph{Erster Bericht.}
\end{center}
\paragraph{}
Moses Chor. 2. 12: „Nachdem er (Artašēs) in Asien vergoldete Bronzestatuen der Artemis, des Herakles und des Apollon gefunden hatte, lässt er sie in unser Land schaffen, um sie in Armavir aufzurichten. Diese nahmen die Hohenpriester, welche vom Geschlechte der Vahnunier waren, und den Apollon und die Artemis stellten sie in Armavir auf. Aber das Männerbildnis des Herakles, welches von Skyllis und Dipoinos dem Kreter angefertigt war, haben sie, da sie ihn für den Vahagn, ihren Genarchen, hielten, in Tarōn, in ihrem erbeigentümlichen Dorfe Aštišat, nach dem Tode des Artašēs aufgestellt.“

Moses Chor. 2. 40: „Aber Eruand, nachdem er seine Stadt gebaut hatte, schaffte Alles von Armavir dorthin ausgenommen die Götteridole... Aber in einer Entfernung von vierzig aspaṙēs von derselben nach Norden baute er nach dem Vorbilde seiner Stadt eine kleine Stadt an dem Flüsse Axurean und nannte sie Bagaran. Eben dort richtete er in derselben die Ordnung der Altäre und schaffte dorthin alle Götteridole aus Armavir. Tempel baute er, und seinen Bruder Eruaz setzte er als Oberpriester ein.“

Moses Chor. 2. 49: „Als Artašēs an den Ort gekommen war, wo der Erasx und der Mecamor sich vereinigen, findet er Gefallen an dem Hügel, baut eine Stadt nach seinem Namen Artašat genannt... er errichtet in derselben einen Tempel und schafft dahin das Bildnis der Artemis aus Bagaran und alle väterlichen Götteridole. Aber Apollons Bildnis richtet er außerhalb der Stadt auf in der Nähe an der (öffentlichen) Straße.“

Ganz anders lautet der zweite Bericht, den wir hier folgen lassen:
\begin{center}
\emph{Zweiter Bericht.}
\end{center}
\paragraph{}
Moses Chor. 2. 12: „Aber nachdem (Artašēs) in Griechenland die Bildnisse des Zeus, der Artemis, der Athena, des Hephaestos, der Aphrodite genommen hat, befiehlt er sie nach Armenien zu transportieren. Noch nicht war es geschehen, dass sie in das Land hineingekommen waren, da hörten sie\footnote{Aus dem Zusammenhang ergibt sich, dass es die mit dem Transport beauftragten Priester sind.} die Nachricht vom Tode des Artašēs. Sie flohen und deponierten die Bildnisse in der Festung von Ani. Und die Priester, sich ihnen widmend, blieben bei ihnen.“

Moses Chor. 2. 14: Als erstes Werk beabsichtigte er Tempel zu erbauen. Aber die Priester, welche aus Griechenland gekommen waren, im Geiste bedenkend, sie möchten in das innerste Armenien verschleppt werden, erdichteten Omina, als ob die Götter daselbst sich niederlassen wollten. Dem zustimmend errichtete Tigran die Statue des olympischen Zeus in der Burg Ani, die der Athena in T͑il, eine andere Statue der Artemis in Erēz und die des Hephaestos in Bagayarinǰ. Aber die Statue der Aphrodite, als der Geliebten des Herakles, befahl er neben der Statue des Herakles selbst an den Opferstätten (= Aštišat) aufzustellen. [Und ergrimmt über die Vahnunier, weil sie gewagt hatten, auf ihrem eignen Erbgute die von seinem Vater übersandte Statue des Herakles aufzustellen, entsetzt er sie ihres Priestertums, und schlägt das Dorf, wo die Statuen aufgerichtet waren, zur königlichen Domäne].\footnote{Das Eingeklammerte gehört dem Inhalte und Zusammenhang nach ganz deutlich in den ersten Bericht.} Nachdem er dergestalt die Tempel erbaut und vor den Tempeln Altäre errichtet hatte, befahl er allen Satrapen Opfer mit Anbetung zu verrichten... (Tigranes) steigt nach Mesopotamien hinab. Und als er hier die Bildsäule des Baršamin gefunden hatte, welche aus Elfenbein, Kristall und Silber angefertigt war, befiehlt er sie fortzuführen und im Flecken T͑ordan aufzustellen.“

Man wird sogleich bemerken, dass wir es mit zwei völlig getrennten Berichten zu tun haben. Der erste lässt die Götterstatuen aus Asien, der zweite aus Griechenland kommen. Der erste Bericht schenkt sein Interesse nur Apollon und Artemis, den Göttern von Artaxata, und Herakles, dem Gotte von Tarōn. Es sind die Tempellegenden dieser beiden Kultusstätten. Interessant ist namentlich die Legende von Aštišat. Hier treffen wir ein mächtiges Priestergeschlecht, die Vahnunier, welche in ihrer \begin{greek}ἱερὰ κώμη\end{greek} den Kult des Vahagn pflegen. König Artašēs hat nach dem Legendenberichte, als er die ersten von griechischen Künstlern gefertigten Statuen in das barbarische Land brachte, im Einverständnis mit den Oberpriestern aus dem Vahnuniergeschlecht gehandelt und sich ihrer als Kommissare bei dem Transporte bedient. Unter Tigranes dagegen kommt es zum völligen Bruch zwischen Thron und Altar. Das taronitische Priesterdorf mit seinen wird von der Krone konfisziert, und die Hohenpriester werden ihrer Würde entkleidet. Das Factum ist nicht unglaubwürdig, da späterhin als Erbherrn von Aštišat eine Nebenlinie des Königshauses auftritt. Dass das hochverehrte Herakles-Vahagnbildnis ein Werk der uralten kretischen Künstler gewesen sei, braucht man nicht notwendig als schriftstellerische Erfindung, sei es des Moses, sei es seiner Quelle hinzustellen. Es ist bekannt, welch starker hellenischer Bildungsstrom in der Tigranesperiode die höheren Schichten der armenischen Bevölkerung ergriffen hat.\footnote{Vgl. Th. Reinach, Mithridate Eupator S. 344 ff.} — König Artavazd der Sohn des Tigranes schriftstellerte griechisch. — Es ist daher nicht unmöglich, dass wir es mit einer alten Tempellegende von Aštišat zu tun haben, welche die Priester und Küster zur Erhöhung der frommen Andacht und der milden Gaben in Umlauf setzten.

Außer der Legende von Aštišat enthält der erste Bericht noch die von Artašat. Die dortigen hochheiligen Gottheiten Anahit (Artemis) und Tiur (Apollon) wurden ursprünglich in Armavir aufgestellt; König Eruand versetzte sie nach Bagaran, Artašēs 2. endlich in seine neu erbaute Residenz Artašat. Richtig und mit Agathangelos, der zuverlässigsten Quelle, in Übereinstimmung steht die Angabe, dass nur Anahit in der Stadt selbst ihren Tempel erhielt, während der Apollotempel außerhalb der Stadt an der großen Heerstraße eingerichtet ward. Der abenteuerliche Bericht von der dreimaligen Verlegung des Sitzes der Götterbilder enthält insofern eine notorische Unrichtigkeit, als er die Gründung von Artašat dem jüngeren statt dem ersten Artašēs zuschreibt. Die ganze Erzählung macht den Eindruck spät und schlecht erfunden zu sein.

Ungleich wertvoller, wenn auch natürlich gleichfalls nicht historisch im strengen Sinne, ist der zweite Bericht. Der Eingang lautet abenteuerlich genug. König Artašēs lässt fünf Götterstatuen aus Griechenland kommen. Allein er stirbt, ehe dieselben den Ort ihrer Bestimmung erreicht haben, und die entsetzten Priester fliehen mitsamt den Gnadenbildern nach Ani. Zutrauen scheint aber jedenfalls der Theil des Berichtes zu verdienen, welcher die tatsächliche Einrichtung der neuen Gottesdienste erst dem Tigranes zuschreibt. Es ist bemerkenswert, dass die in diesem Berichte erwähnten Heiligtümer, von denen Ani in Daranałi, Erēz und T͑il in Akilisene und Bagayarinǰ in Derxene liegen, sämtlich den Reichsteilen angehören, welche erst Artaxias und seine Nachfolger bis auf Tigranes Großarmenien einverleibt haben. Strabo 11. 528 C. Es passt durchaus in das Zeitalter der Eukratidas, Menandros und Apollodotos, dass ein unabhängig gewordener Seleukidenstatthalter und sein mächtiger Nachfolger die hellenischen Götterdienste in derselben Weise in ihren asiatischen Barbarenstaat importieren und mit den einheimischen Diensten verschmelzen, wie etwa Antiochos Epiphanes dies in Palästina versuchte. Wie dort der Gott von Jerusalem als \begin{greek}Ζεὺς Ὀλύμπιος\end{greek} und der von Samaria als \begin{greek}Ζεὺς Ξένιος\end{greek} erscheinen, so müssen sich hier Aramazd, Anahit, Mihr u. s. f. gefallen lassen mit den neuen Namen \begin{greek}Ζεὺς Ὀλύμπιος, Ἄρτεμις, Ἥφαιστος\end{greek} geschmückt und durch griechische Künstlerhand in Metall ausgegossen zu werden. Dass man griechische Priester zum Tempeldienst nach Armenien schleppte, stimmt mit dem, was wir aus griechischen Quellen über Tigranes' politisches System wissen, durchaus überein. Eine echte Tempellegende ist sodann die Nachricht, wonach die Priester göttliche Stimmen vernahmen, welche den Willen der Himmlischen kundgaben, sich an bestimmten Orten zu fixieren. Der Bericht sieht in diesen Omina nur ein Vorgeben der Priesterschaft; dabei ist es nicht nötig eine christliche Korrektur anzunehmen; ein euhemeristischer Heide konnte gerade so lehren. Genau, wie diese griechischen Götter, lässt Tigranes auch ein Bild des syrischen Himmelsgottes nach T͑ordan in Daranałi schaffen, und es ist bemerkenswert dass die schaffende Kraft der Volksmythologie sich dieser Gestalt bemächtigt hat. Wir sehen demnach, dass das religiöse Volksbewusstsein eine mächtige Anregung dieser uns recht gewöhnlich und schablonenhaft erscheinenden Maßregel eines asiatischen Despoten verdankt. Die Genesis religiöser Ideen, welche sich die Späteren oft sehr erhaben und geheimnisvoll denken, ist eben oft von ziemlich profanen und zufälligen Umständen abhängig. Unserem Bewusstsein erscheint eine religiöse Propaganda mit so äußerlichen Mitteln fast undenkbar, und doch sind die Belege für parallele Vorgänge in der Religionsgeschichte nicht selten; man denke nur an das so folgenreiche Dekret des Artaxerxes 2 Mnemon über den Anahitkultus oder an die Bekehrung der Russen zum Christentum.

Eine im Grunde müßig Frage ist die, ob der Kultus der Armenier vor dem Einfluten der hellenistischen Zivilisation ein vollkommen bildloser gewesen sei. Einige unförmliche oder symbolische Gnadenbilder mögen schon vorher die Andacht belebt haben. Jedenfalls haben wir uns aber an die Tatsache der Überlieferung zu halten, dass die philhellenischen Könige und Reichsgründer Artaxias und Tigranes, wie sie auch sonst die griechische Kultur im Reiche mächtig förderten, so auch den religiösen Bilderdienst, für primitive Völker eines der mächtigsten und wirksamsten Bildungsmittel, in Armenien eingeführt haben. Dass der neue Bilderdienst sich widerstandslos durchsetzen ließ, ist nicht eben wahrscheinlich; allein Moses' Erzählung von der Opposition der Bagratunier, mit ihrer angeblich jüdischen Herkunft zusammenhängend, kann nach Gutschmids Ausführungen hierfür nicht verwertet werden.

Es ist nun gewiss nicht zufällig, dass „die spärlichen, aber umso wertvolleren Notizen über die Geschichte des armenischen Heidentums“ bei Moses da einsetzen, wo der Bericht des Mar Abas Katinā sein Ende erreicht hat. Bereits Gutschmid hat es als Möglichkeit ausgesprochen, dass diese Berichte der Tempelchronik des Olympios von Ani entstammten.\footnote{Kl. Schr. 3. S. 328.} Moses (2. 48) erwähnt diesen Autor als \begin{armenian}Ուղիւպ քուրմ Հանւոյ\end{armenian} Ułiup Priester von Ani und sein Werk als \begin{armenian}մեհենական պատմութիւնք\end{armenian} \begin{greek}ἱερατικαὶ ἱστορίαι\end{greek} Tempelgeschichten. Daneben führt er auch (2. 66) die singularische Form \begin{armenian}մեհենական պատմութիւն\end{armenian} Tempelgeschichte und \begin{armenian}մեհենիցն պաշտօնք\end{armenian} Tempelriten an. An der ersten Stelle führt er das Werk für ein Ereignis der Geschichte des Königs Artašes an. An der zweiten bemerkt er ausdrücklich, dass in der Tempelgeschichte „die Taten der Könige“ eingezeichnet waren, ferner der Gnostiker Bardesanes (Bardacan) habe das Werk fortgesetzt von König Artavazd bis auf Chosrov. Das ganze Werk hat dann der letztere ins Syrische übersetzt. Später ist dann dieses Werk — er berichtet nicht durch wen — ins Griechische übersetzt worden. Dabei sagt Moses nicht ausdrücklich, in welcher Sprache Olympios selbst geschrieben habe. Da das Werk erst ins Syrische, dann ins Griechische übersetzt worden ist, denkt der Leser unwillkürlich, das Original werde wohl armenisch gewiesen sein, und sollte nicht Moses bewusster Weise diesen Schein haben erwecken wollen? Diese Übersetzungen sind allemal bei Moses hochbedenklich. Man denke an Mar Abas Katinā, der ein aus dem Chaldaeischen ins Griechische übersetztes Werk vorfand und daraus einen Auszug in griechischer und syrischer Schrift macht. Tatsächlich hat einfach ein syrisches Werk vorgelegen. Die Autorschaft des Bardesanes unterliegt gerechten Bedenken.\footnote{Gutschmid, Kl. Schr. 3. S. 304.} Sie ist wohl nur dem Bedürfnis des Moses entsprungen, einen christlichen schützenden Gewährsmann für seine heidnischen Tempelgeschichten vorzuschieben. Dagegen die Tempelgeschichte des Olympios selbst ins Reich der Erfindungen zu verweisen, scheint mir kein Grund vorzuliegen.\footnote{Wie J. Marquart ZDMG 49. S. 656 tut, welcher in Olympios nur eine Graecisierung des Oberpriesters Mažan sieht. Indessen Erfindung fabelhafter Geschichtswerke ist mehr die Art des Zenob von Glak als des Moses. Dieser zitiert gewöhnlich vorzügliche Autoritäten (Africanus, Firmilianus, Ariston von Pella, Phlegon u. s. f.) nur freilich fast immer für Dinge, die sie nicht berichtet haben oder nicht berichtet haben können. Letzteres könnte auch bei Olympios zutreffen, ohne dass daraus folgt, dass seine Persönlichkeit ebenfalls erdichtet sei.}

Neben Olympios zitiert Moses 2. 48 für die Geschichtsepoche, wo sein alter Führer Mar Abas versagt, die Bücher der Perser und die Gesänge der Vipasank͑. Bei den ersteren ist wohl an das Werk des Barsuma (Choṙohbut) zu denken.\footnote{Moses 2. 69, 70 vgl. von Gutschmid, Kl. Schr. 3. S. 301 ff.} Aber ein großer Theil seines Geschichtsstoffes kann weder einer Geschichte der Parther, noch den historischen Liedern entstammen; er muss auf ein Geschichtswerk armenischen Ursprungs zurückgehen. Nun finden sich in den einschlägigen Berichten über die Könige Artašēs, Tigran, Eruand u. s. f. zahlreiche Angaben über Tempel und Kulteinrichtungen; da hat doch Gutschmids Vermutung alle Wahrscheinlichkeit für sich, dass dieselben Olympios' Werk entstammen. Wie hat Olympios' geschrieben? griechisch oder syrisch? Mit Recht hat Gutschmid darauf aufmerksam gemacht, dass die seinem Werke entstammenden Namen keine Entlehnung aus dem Griechischen verrieten. Das Werk war wohl syrisch abgefasst, und darum musste Bardesanes dasselbe fortsetzen, übersetzen und unter seinen christlichen Schutz nehmen.

Die Tempellegenden zeigen eine euhemeristische Deutungsweise, so bezüglich der Omina, welche den Götterstatuen ihren Platz anwiesen und bezüglich des Vanaturtempels von Bagavan, dass ein Priestergrab sein soll. Es liegt keinerlei Nötigung vor, diese Rationalisierung erst den Christen zuzuschreiben; der Verfasser der Tempelgeschichte hat sich eben nach der Art eines Philon von Byblos oder Suetonius an die Gebildeten gewandt, welche längstens derartige Deutungen der alten Mythen kannten.
\clearpage
\section{Armenische Theologie}
\paragraph{}
Spuren einer Systematisierung des einheimischen Götterglaubens, Theologumena der armenischen Priesterschaft, lassen sich nur in geringer Zahl nachweisen. Wie anderwärts finden wir auch in Armenien die Götter in Triaden und Ogdoaden geordnet. In Südarmenien in Tarōn ist uns die Trias von Aštišat bekannt: Vahagn — Anahit — Astłik. Eine zweite Trias begegnet uns in Tiridates' Edikt, woraus sich ergibt, dass diese Götter offiziell als die drei großen Schutzgottheiten des Landes anerkannt waren Agathang. S. 82: „Trdat der König von Großarmenien an die Megistanen (\begin{armenian}մեծամեծս\end{armenian}), die Fürsten, die Satrapen, die Beamten und die übrige Menschheit, die ihr in Burgen, Flecken, Dörfern und auf dem platten Lande wohnt, an die Freien und Bauern, Allen miteinander Heil! Heil und Wohlstand komme durch der Götter Hülfe, Fülle des Überflusses vom starken Aramazd, Schutz von der Herrin Anahit, Tapferkeit vom tapferen Vahagn zu dem gesamten Reiche der Armenier u. s. f.“

Aber auch die Spuren eines alle größeren Gottheiten Armeniens umfassenden Systems lassen sich nachweisen. Natürlich waren auch hier die Priester bestrebt, ein die sämtlichen Himmelsbewohner umfassendes System, so gut wie ihre Genossen in Ägypten, Babylon oder Athen zu konstruieren Agathangelos berichtet, dass der König Chosrov, der Vater Trdats nach seinem Siege über die Perser ein großes Opfer bei den sieben Altären der Tempel veranstaltete S. 26: „Da gab er Befehl nach den verschiedenen Gegenden Boten zu entsenden und Briefe zu verfassen, an den sieben Altären der Tempel Weihegaben darzubringen den Bildern der Götteridole: mit weißen Stieren, weißen Ziegen, weißen Pferden und weißen Mauleseln, mit goldenen und silbernen Schmucksachen mit glitzernden Fransen, mit Seidenzeug, mit Fransen und Borten geschmückt, mit goldenen Kronen und silbernen Gerätschaften, mit kostbaren Edelsteinen, mit Gold und Silber (geschmückten) Gefäßen, mit prachtvollen Gewändern und köstlichen Schmucksachen, ehrte er die Stätten des erbeigentümlichen Kultus seines Geschlechtes, der Aršakunier.“

Die Kultusstätte der sieben Altäre kann nicht, wie Tommaseo meint, der in Moses Geographie (p. 33 Soukry)\footnote{Vgl. Inčičean, Beschreibung des alten Armeniens S. 326.} erwähnte Gau von P͑aytakaran \begin{armenian}Եւթն փորակեան բագինք\end{armenian}, der Altar der sieben Nischen (?) sein; denn unmittelbar vor dem Berichte von dem Opfer wird erzählt, dass der König glanzvoll in seine Hauptstadt Vałaršapat eingezogen sei. Von dorther erlässt er den Befehl an sämtliche Provinzen, zum Nationaldankfest zusammenzukommen. Das Zentralheiligtum der 7 Altäre muss also in oder bei Vałaršapat, im Mittelpunkt des Reichs, nicht im entfernten Osten sich befunden haben. Die sieben Altäre erinnern an die sieben Amescha spenta, an die Abdarda der taurischen Alanen (Tomaschek bei Wissowa-Pauly unter Abdarda) u. s. f. Das Opfer von Massen hellfarbiger Tiere entspricht den Riten des Awesta.\footnote{W. Geiger, ostiranische Kultur S. 469.} Damit stimmt die ausdrückliche Angabe, dass der König den Opferritus nach der Weise der Väter, der parthischen Arsaciden, vollzieht. Möglicherweise mit dieser Siebenzahl im Zusammenhang steht die Achtzahl, von der wir allerdings nur eine, aber wertvolle und authentische Spur haben. Vahagn heißt in seinem Kultorte Aštišat \begin{armenian}ութերորդ պաշտօն\end{armenian} \begin{greek}ὄγδοον σέβασμα\end{greek}. Auf iranischem Boden kann ich eine Betonung der Achtzahl nicht finden; wohl aber kennt wenigstens die späte phönizische Theologie dieselbe. Darüber berichtet Damaskios (Photius bibl. 242 p. 352b Bekker): \begin{greek}ὅτι ὁ ἐν Βηρυτῷ, φησίν, Ἀσκληπιὸς οὐκ ἔστιν Ἕλλην, οὐδὲ Αἰγύπτιος, ἀλλά τις ἐπιχώριος Φοῖνιξ. Σαδύκῳ γὰρ ἐγένοντο παῖδες, οὓς Διοσκούρους ἑρμηνεύουσι καὶ Καβείρους. ὄγδοος δὲ ἐγένετο ἐπὶ τούτοις ὁ Ἔσμουνος, ὃν Ἀσκληπιὸν ἑρμηνεύουσιν\end{greek}. Wir haben also hier eine Heptas von Gottheiten, Söhne eines Vaters und neben ihnen den Achten. Ohne Frage haben wir es mit recht später Theologie zu tun, etwa in der Art des Philo Byblius. Indessen bei dem starken Einfluss, welchen notorisch die hellenistischen Syrer auf Armenien ausübten, ist der Gedanke nicht von vornherein abzuweisen, auch die armenische Theologie sei durch diese oder ähnliche Konstruktionen beeinflusst worden. Wenn wir nämlich die Götter zusammenstellen, deren Tempel durch Tiridates und Gregor zerstört wurden und die demnach fragelos die vom Staate anerkannten Kulte repräsentieren, so ergeben sich außer Aramazd, dem Vater aller Götter, gerade acht Gottheiten.

\begin{enumerate}
    \item Vanatur dessen Fest am ersten Navasard gefeiert wird; der also wahrscheinlich, wie den Festkalender, auch den Götterreigen eröffnet.

    \item Mihr, der Sohn des Aramazd.

    \item Anahit, die Goldmutter die Tochter des Aramazd.

    \item Nanēa, die Tochter des Aramazd.

    \item Baršamin, der weißglänzende Gott.

    \item Astłik, die Geliebte Vahagns.

    \item Tiur, der Schreiber des Ormizd.

    \item Vahagn der achte.
\end{enumerate}
\paragraph{}
Die spätere Theologie hat wohl auch die nach Nanēa folgenden drei Gottheiten als Kinder des Aramazd angesehen, wenn dies auch nicht bezeugt ist; denn nur so rechtfertigt sich die Benennung des Aramazd als Vaters aller Götter. Natürlich muss eine solche Herstellung des armenischen Pantheons stets Vermutung bleiben. Indessen die Parallelen mit dem semitischen Mythologemen: der Göttervater, die Sieben, der Achte sind so auffällig, dass man wohl schwerlich nur an Zufall denken kann.

Endlich ist noch ein allerdings gänzlich isoliertes Bruchstück mythologischer Gelehrsamkeit zu erwähnen. Moses von Chorēn gedenkt 1. 31 der Ansicht einiger, es gebe vier oder mehr Aramazd, „deren einer auch ein gewisser Kund Aramazd ist.“ Die Unterscheidung verschiedener Zeus, Herakles u. s. f., welche philosophierende Mythologen aufgestellt haben, ist demnach von den Armeniern den Griechen entlehnt worden. Aber während die griechischen Theologen stets drei Zeus kennen,\footnote{Vgl. Cicero de nat. deor. 3. 21. Ampelius 9. Clement. Alexandr. Protrept. p. 8 Potler u. s. f.} werden hier vier und noch mehr erwähnt. Clemens, nachdem er die drei Zeus aufgezählt hat, kommt an einer späteren Stelle (p. 33 P.) noch einmal auf eigenthümliche Zeusgestalten: \begin{greek}οὐχὶ μέντοι Ζεὺς φαλακρὸς ἐν Ἄργει, τιμωρὸς δὲ ἄλλος ἐν Κύπρῳ τετίμησθον\end{greek}; Kund Aramazd ist nun nichts, als Übersetzung von Zeus Phalakros. Es würde ganz mit der willkürlichen, die Zeugnisse etwas sich zurechtlegenden Art des Moses übereinstimmen, wenn er aus diesen beiden Angaben des Clemens seine Notiz von den vier und mehr Aramazd kombiniert hätte. Auf alle Fälle handelt es sich hier lediglich um einen Einfall eines des Griechischen kundigen Gelehrten, nicht um graecisirende Vorstellungen, die beim Volk oder wenigstens bei den höheren Volksschichten Eingang gefunden hatten.

Das Resultat, welches man aus einer eingehenden Betrachtung des armenischen Pantheons gewinnen kann, ist nicht ganz unerheblich. Es wirft bedeutsame Schlaglichter auf Völker, welche in analogen Verhältnissen leben d. h. welche auf derselben Kulturstufe, wie die heidnischen Armenier stehen, also Analphabeten und literaturlos sind. Ein solches barbarisches Volk waren nun zweifellos nicht nur die Hellenen von Mykene, Tiryns und Orchomenos, sondern auch die Kulturstufe, welche sie im 9 und 8 Jahrhundert erreicht hatten, war der armenischen des 2 und 3 nachchristlichen Jahrhunderts ähnlich. Ein solches Volk, wenn es auch längst Ackerbau treibt, Wein- und Ölbau pflegt, Goldschmuck von asiatischen Händlern eintauscht, ist darum doch noch ein geistiges Kind, ein weißes Blatt, Wachs, welches widerstandslos jeden Eindruck von außen in sich aufnimmt. Schön und treffend sagt darum Herodot 1. 135 von den kulturell auf gleicher Stufe stehenden Persern: „Fremde Bräuche nehmen die Perser von allen Menschen am leichtesten an.“ Die Armenier sind durch Geschichte, Königshaus und Adel aufs engste mit den iranischen Parthern verknüpft; sie wurden die Nachbarn der hochgebildeten Syrer. Was ist da wunderbares, dass ihr Götterhimmel iranisiert und semitisiert wurde! Man sollte meinen, diese mächtigen Wogen fremder Gesittung hätten über der kleinen, kraft ihrer niedrigen Kultur wenig widerstandsfähigen Nation zusammengeschlagen und ihre noch schwach entwickelte Individualität völlig erstickt. Das Gegenteil ist der Fall. Eine spezifisch nationale Gottheit, wie Vanatur, gehört tatsächlich zu den bedeutendsten der ganzen Nation; sein Heiligtum bleibt ein Zentralwallfahrtsort des gesamten Volkes bis in die christliche Epoche. Ursprünglich fremde Gestalten, wie der iranische Vahagn, werden so vollständig armenisiert und in die nationale Art des Volkes umgeschmolzen, dass die rechtgläubigen Mazdayasnier in den so vieles aus Iran entlehnenden Armeniern nur Götzendiener erkennen wollen. Die nationale Eigenart ist demnach durch die fremde Einwirkung nicht aufgesogen und unterdrückt, sondern gerade erst zu wahrer, selbständiger Entfaltung gekommen. Die Konsequenzen für die griechische Götterlehre ergeben sich hieraus von selbst.
\clearpage
\end{document}
